\subsection{Contention Resolution in sistemi distribuiti}
Dai $n$ processi $P_1,\dots,P_n$ ognuno dei quali compete per l'accesso a un database condiviso. Se due o più processi accedono al database contemporaneamente, tutti i processi sono bloccati. Determinare un protocollo per assicurare a ogni processo l'accesso al database. \emph{Nota: I processi non possono comunicare tra di loro}.

\vspace{1em}\noindent
\emph{Protocollo:} Ogni processo richiede l'accesso al database al tempo $t$ con probabilità $p = \frac{1}{n}$.

Si definisce $S_{i,t}$ l'evento in cui il processo \emph{i-esimo} riesce ad accedere al database al tempo $t$. 
\begin{theorem}
    Siano $S_{i,t}$ le variabili aleatorie definite come sopra
    $\frac{1}{en} \leq \Pr[S_{i,t}] \leq \frac{1}{2n}$.
\end{theorem}

\begin{proof}
    Ogni processo è indipendente, per cui per l'indipendenza $\Pr[S_{i,t}] = p(1-p)^{n-1}$, in cui $p$ è la probabilità di successo di un solo processo, e $(1-p)^{n-1}$ è la probabilità in cui nessuno degli altri $n-1$ processi richiede l'accesso.

    \vspace{1em}
    Per cui impostando $p=\frac{1}{n}$ si ha che 
    \[
        \Pr\left[S_{i,t}\right] = \frac{1}{n}\left(1-\frac{1}{n}\right)^{n-1} \approx \boxed{\frac{1}{en}}
    \]
    \emph{Nota:} $\left( 1 - \frac{1}{n}\right)^{n-1}$ converge a $\frac{1}{e}$ per $n \to \infty$.
\end{proof}
% TODO: Riformulare
\begin{theorem}
    La probabilità che l'evento \emph{i-esimo} fallisca ad accedere al database in $en$ tentativi $\leq \frac{1}{e}$. Dopo $en(c \log n)$ tentativi, la probabilità $\leq \frac{1}{n^c}$
\end{theorem}
\begin{proof}
    Sia $F_{i,t}$ l'evento in cui il processo $i$ fallisce fallisce nell'accesso al database nei round $1,\dots,t$. Poiché i tentativi sono indipendenti tra di loro, si ha che 
    \[
        \Pr[F_{i,t}] \leq \left(1- \frac{1}{en}\right)^t
        \]
        Per cui scegliendo $t = \lceil en \rceil$ si ha 
        \[
            \Pr[F_{i,t}] \leq \left(1-\frac{1}{en}\right)^{\lceil en \rceil} \leq \left(1-\frac{1}{en}\right)^{en} \leq \frac{1}{e}
            \]
            Scegliendo $t = \lceil en \rceil\cdot\lceil c \log n \rceil$ si ha 
            \[
                \Pr[F_{i,t}] \leq \left(\frac{1}{e}\right)^{c \log n} = \frac{1}{n^c}
            \]
\end{proof}
Dati questi teoremi come basi, è possibile definire la probabilità per cui tutti i processi abbiano successo in $2e\, n \log n$ tentativi
$\Pr[\text{Successo}] \geq 1 - \frac{1}{n}$

\begin{proof}
    Sia $F_t$ l'evento in cui almeno uno degli $n$ processi fallisce nell'accesso nei tentativi $1,\dots,t$
    \begin{align*}
        \Pr[F_t] &= \Pr\left[ \bigcup_{i=1}^{n} F_{i,t} \right] \leq\\
        (\substack{\text{Per \emph{union buond \ref{union}}}}) &\leq \sum_{i=1}^n \Pr[F_{i,t}] \leq n\left(1-\frac{1}{en}\right)^t
    \end{align*}
    Inoltre per $t = 2e \, n \log n$ tentativi ($c = 2$), per il teorema precedente
    \begin{align*}
        \Pr[F_{i,t}] \leq n \cdot \left(\frac{1}{e}\right)^{2 \log n} = \frac{1}{n^2}.
    \end{align*}
    Concludendo \[
        \Pr[F_t] \leq n \cdot \frac{1}{n^2} = \frac{1}{n}
    \]
\end{proof}
% TODO: Rifinire
\subsection{Problema del Load balancing}
Si ha un sistema in cui $m$ \emph{jobs} arrivano in stream e devono essere processati immediatamente su $n$ processori identici. È necessario trovare un assegnamento che bilanci il \emph{carico di lavoro} tra i processori.

Un approccio semplice e deterministico è quello di avere un \emph{controller} centralizzato che distribuisca il carico con uno \emph{scheduling} Round-Robin. Ogni processore riceve al massimo $\lceil \frac{m}{n} \rceil$ \emph{jobs}. 

Per un approccio decentralizzato si fa uso della probabilità e del calcolo probabilistico, ogni \emph{job} viene assegnato a un processore in modo uniformemente casuale. Si calcola la probabilità che un processore riceva troppi \emph{jobs} $\left(> \lceil \frac{m}{n} \rceil \right)$.

A tale scopo si definiscono le variabili aleatorie $X_i$ il numero di \emph{jobs} assegnati al processore $i$,
\[
    Y_{ij} = \begin{cases*}
        1 \text{ se il job $j$ è assegnato a $i$}\\
        0 \text{ altimenti}
    \end{cases*}
\]

Per cui si hanno i dati $\E{Y_{ij}} = \frac{1}{n}, \Pr[Y_{ij} = 1] = \frac{1}{n}$. Inoltre $X_i = \sum{j=1}^{n}Y_{ij}$ e si definisca $\mu = \E{X_i} = 1$, rappresenta il carico \emph{atteso} del processore $i$.

Applicando i Chernoff Bounds da \ref{chernoff}, con $\delta = c - 1$ si ha
\[
    \Pr[X_i > c] < \frac{e^{c-1}}{c^c}
\]

Sia $\gamma(n)$ il numero $x$ tale che $x^x = n \implies (\gamma(n)^{\gamma(n)} = n)$ e sia $c = e\cdot \gamma(n)$
\begin{align*}
    &\Pr[X_i > c] < \frac{e^{c-1}}{c^c} < \left( \frac{e}{c}\right)^{c} =\\
    &=\left( \frac{1}{\gamma(n)}\right)^{e\gamma(n)} < \left( \frac{1}{\gamma(n)} \right)^{2\gamma(n)} = \frac{1}{n^2}.
\end{align*}

Per lo union bound \ref{union} si mostra il risultato
\[
    \Pr\left[\bigcup_{j = 1}^{n} Y_{ij}\right] < \Pr\left[{\sum_{j=1}^{n}Y_{ij}}\right] = \Pr[X_i] > \frac{1}{n^2}
\]
Per cui $\lnot\Pr{\left[\sum_{i = 1}^nX_i\right]} = 1 - n\Pr\left[ \bar{X}_i  \right] < 1 - \frac{1}{n}$ è la probabilità che \emph{nessun processore riceve più di $e\cdot\gamma(n)$ jobs}. $e\cdot\gamma(n) = \Theta\left(\frac{\log n}{\log \log n} \right)$.

\begin{theorem}
    Si supponga il numero di \emph{jobs} $m = 16 n \log n$. Allora in media ognuno degli $n$ processori gestisce $\mu = 16 \log n$ \emph{jobs}, con alta probabilità ogni processore gestirà tra $x$ compreso tra $\frac{1}{2}$ e $2$ carico medio.
\end{theorem}

\begin{proof}
    Siano $X_{i}, Y_{ij}$ variabili aleatorie definite come sopra, applicando il Chernoff bound con $\delta = 1$
    \begin{align*}
        &\Pr[X_i > 2\mu] < \left( \frac{e}{4}\right)^{16 n \log n} < \left( \frac{1}{e}\right)^{\log n} = \frac{1}{n^2}\\
        &\Pr\left[X_i > \frac{1}{2}\mu\right] < e^{-\frac{1}{2}\left(\frac{1}{2}\right)^2{16 n \log n}} = \frac{1}{n^2} 
    \end{align*}
    Per lo union bound \ref{union} ogni processore ha il carico tra $\frac{1}{2}$ e $2$ del carico medio, con probabilità $\geq 1- \frac{2}{n}$
\end{proof}