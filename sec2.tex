\subsection{Contention Resolution in sistemi distribuiti}
Dai $n$ processi $P_1,\dots,P_n$ ogniuno dei quali compete per l'accesso a un database condiviso. Se due o più processi accedono al database contemporaneamente, tutti i processi sono bloccati. Determinare un protocollo per assicurare a ogni processo l'accesso al database. \emph{Nota: I processi non possono comunicare tra di loro}.

\vspace{1em}\noindent
\emph{Protocollo:} Ogni processo richiede l'accesso al database al tempo $t$ con probabilità $p = \frac{1}{n}$.

Si definisce $S_{i,t}$ l'evento in cui il processo \emph{i-esimo} riesce ad accedere al database al tempo $t$. 
\begin{theorem}
    Siano $S_{i,t}$ le variabili aleatorie definite come sopra
    $\frac{1}{en} \leq \Pr[S_{i,t}] \leq \frac{1}{2n}$.
\end{theorem}

\begin{proof}
    Ogni processo è indipendente, per cui per l'indipendenza $\Pr[S_{i,t}] = p(1-p)^{n-1}$, in cui $p$ è la probabilità di successo di un solo processo, e $(1-p)^{n-1}$ è la probabilità in cui nessuno degli altri $n-1$ processi richiede l'accesso.

    \vspace{1em}
    Per cui impostando $p=\frac{1}{n}$ si ha che 
    \[
        \Pr\left[S_{i,t}\right] = \frac{1}{n}\left(1-\frac{1}{n}\right)^{n-1} \approx \boxed{\frac{1}{en}}
    \]
    \emph{Nota:} $\left( 1 - \frac{1}{n}\right)^{n-1}$ converge a $\frac{1}{e}$ per $n \to \infty$.
\end{proof}
% TODO: Riformulare
\begin{theorem}
    La probabilità che l'evento \emph{i-esimo} fallisca ad accedere al database in $en$ tentativi $\leq \frac{1}{e}$. Dopo $en(c \log n)$ tentativi, la probabilità $\leq \frac{1}{n^c}$
\end{theorem}
\begin{proof}
    Sia $F_{i,t}$ l'evento in cui il processo $i$ fallisce fallisce nell'accesso al database nei round $1,\dots,t$. Poiché i tentativi sono indipendenti tra di loro, si ha che 
    \[
        \Pr[F_{i,t}] \leq \left(1- \frac{1}{en}\right)^t
        \]
        Per cui scegliendo $t = \lceil en \rceil$ si ha 
        \[
            \Pr[F_{i,t}] \leq \left(1-\frac{1}{en}\right)^{\lceil en \rceil} \leq \left(1-\frac{1}{en}\right)^{en} \leq \frac{1}{e}
            \]
            Scegliendo $t = \lceil en \rceil\cdot\lceil c \log n \rceil$ si ha 
            \[
                \Pr[F_{i,t}] \leq \left(\frac{1}{e}\right)^{c \log n} = \frac{1}{n^c}
            \]
\end{proof}
Dati questi teoremi come basi, è possibile definire la probabilità per cui tutti i processi abbiano successo in $2e\, n \log n$ tentativi
$\Pr[\text{Successo}] \geq 1 - \frac{1}{n}$

\begin{proof}
    Sia $F_t$ l'evento in cui almeno uno degli $n$ processi fallisce nell'accesso nei tentativi $1,\dots,t$
    \begin{align*}
        \Pr[F_t] &= \Pr\left[ \bigcup_{i=1}^{n} F_{i,t} \right] \leq\\
        (\substack{\text{Per \emph{union buond \ref{union}}}}) &\leq \sum_{i=1}^n \Pr[F_{i,t}] \leq n\left(1-\frac{1}{en}\right)^t
    \end{align*}
    Inoltre per $t = 2e \, n \log n$ tentativi ($c = 2$), per il teorema precedente
    \begin{align*}
        \Pr[F_{i,t}] \leq n \cdot \left(\frac{1}{e}\right)^{2 \log n} = \frac{1}{n^2}.
    \end{align*}
    Concludendo \[
        \Pr[F_t] \leq n \cdot \frac{1}{n^2} = \frac{1}{n}
    \]
\end{proof}
% TODO: Rifinire
