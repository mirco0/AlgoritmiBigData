In questa sezione si parla di problemi per cui la dimensione dei dati è eccessiva, ad esempio si potrebbe essere interessati a valutare uno dei seguenti problemi
\begin{enumerate}
    \item \textit{Pagine con una grande porzione di parole simili}
    \item \textit{Clienti con acquisti simili}
    \item \textit{Immagini con elementi silimi}
\end{enumerate}
\begin{example}
    Input: punti ad alte dimensioni $x_1,x_2,\dots,x_N$
    \[
        \begin{bmatrix}
            1 & 4 & 1\\
            0 & 2 & 1\\
            0 & 1 & 0\\
        \end{bmatrix}
        \implies
        \begin{bmatrix}
            1 & 4 & 1 & 0 & 2 & 1 & 0 & 1 & 0
        \end{bmatrix} \in \{ 0,1,\dots,c\}^h.
    \] per $h$ molto grande
    Si introduce una funzione di distanza $d(x_1,x_2)$ che quantifichi quanto i due dati sono \emph{vicini}.
    Un'applicazione comune consiste nel trovare tutte le coppie di dati in un insieme di dati che sono ``abbastanza'' vicini (rispetto a una soglia $s$) per cui $d(x_1,x_2) \leq s$.

    Una soluzione deterministica banale richiede tempo $O(N^2\cdot h)$ con $N$ numero di punti e $h$ dimensione dei dati. Banalmente si calcola la distanza per ogni coppia possibile $\binom{n}{2} = O (n^2)$.
\end{example}

È possibile trovare coppie di dati simili in tempo $O(N \cdot h')$ con $h' << h$, come spiegato di seguito.
\subsection{Document Similarity}
Dato un grande insieme di documenti $U^*$ grande $N \approx 10^9$, si vogliono trovare i documenti simili. Le applicazioni possibili riguardano la ricerca di siti web duplicati e la ricerca di notizie simili.

Il problema descritto presenta diverse sfide 
\begin{itemize}
    \item molti pezzi piccoli di un documento possono essere in ordine diverso
    \item il numero di documenti ($N$) è troppo elevato per confrontare tutte le coppie
    \item i documenti possono essere talmente grandi ($h$) da non poter essere mantenuti in memoria 
\end{itemize}

È inoltre necessario definire una funzione di distanza adeguata, a tale scopo si fa uso della \emph{Jaccard Similarity} definita su due insiemi $C_1,C_2$ come segue
\[
    \text{Jaccard similarity}(C_1,C_2) = \frac{|C_1 \cap C_2|}{|C_1 \cup C_2|}
\]
Si definisce anche un'altra misura \emph{Jaccard Distance} $d(C_1,C_2) = 1- \jsim(C_1,C_2)$. Resta ora da definire come applicare la Jaccard Similarity a due documenti (stringhe di lunghezza finita). 

Si definiscono i punti chiave della risoluzione del problema
\begin{enumerate}
    \item Input: Un universo $U$ enorme di documenti
    \item Shingling: conversione dei documenti in \emph{grandi insiemi}
    \item Min-Hashing: conversione di grandi insiemi in \emph{piccole firme} mantenendo la $\jsim$
    \item Local-Sensitive-Hashing: Individuare le firme che potrebbero essere simili 
    \item Output: coppie di documenti candidati 
\end{enumerate}

Restano ora da definire nel dettaglio le tecniche descritte
\subsubsection{Shingling}
Facciamo uso della tecnica di \emph{Shingling} per convertire i documenti in insiemi. Un \emph{k-shingle} per un documento è una sequenza di $k$ token che appaiono in un documento. I \emph{token} possono generalmente essere definiti come: caratteri o parole, si fa uso di $U_t$ per rappresentare l'insieme di tutti i possibili \emph{token}.
\begin{example} Si assumano i token come caratteri dell'alfabeto. Sia $k=2$
    
    Per i documenti $D_1 = abcab, D_2 = ccadf$, gli insiemi di \emph{2-shingle} corrispondenti sono 
    \[
        S(D_1) = \{ab,bc,ca\} \quad S(D_1) = \{cc,ca,ad,df\}
    \]
    
\end{example}

Ogni documento $D \in U$ può quindi essere rappresentato come l' insieme dei suoi \emph{k-shingle} $C = S(D)$. In modo più pratico ogni documento è rappresentato come $C=S(D)$ un vettore binario avente ogni possibile elemento dell'insieme $U^k$. Codificando gli insiemi come vettori binari è possibile definire l'intersezione come \emph{bitwise and} e l'unione come \emph{bitwise or}.È facile notare che questi vettori sono molto grandi, ma conservano i dati in modo sparso.

\vspace{1em}
È inoltre facile rappresentare la $\jsim$ con un'interpretazione probabilistica
\[
    \jsim(C_1, C_2) = \frac{C_1 \cap C_2}{C_1 \cup C_2} = \Pr[C_1 \cap C_2 | C_1 \cup C_2].
\]

Concludendo i \emph{k-shingle} permettono di rappresentare il contenuto dei documenti attraverso degli elementi di un insieme. Questa tecnica permette inoltre di trovare documenti simili anche per testi che appaiono in ordine diverso.

L'insieme di tutti i dati gestiti è visibile come una matrice $M \in {0,1}^{N\times m}$ con $N$ numero di documenti e $m = |U|^k$ \emph{shingles} possibili. Si presenta un esempio di seguito
%TODO: disegnare tabella pagina 30 Ch03-ABD-Andy_SLIDES 3.pdf

\vspace{1em}\noindent
Si passa ora alla tecnica del Min-hashing che permette di rendere i grandi insiemi di \emph{k-shingles} in piccole firme mantenendo la misura di similarità
\subsubsection{Min-hashing}
Si presenta un esempio per mostrare perché il min-hashing è necessario
\begin{example}
    Sia $N = 10^6$ il numero di documenti, per cui si hanno $10^6$ \emph{k-shingles}, e si ottengono $\frac{(10^6)^2}{2} = 5\cdot10^{11}$ confronti.
\end{example}
La tecnica precedente permette solo una rappresentazione robusta dei documenti, con il min-hashing si cerca un modo per trovare documenti simili tramtite delle piccole firme. 

La difficoltà consiste nel definire una firma che conservi la similarità tra due insiemi.
