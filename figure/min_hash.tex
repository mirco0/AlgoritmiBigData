\begin{figure}[ht]
\centering
\begin{tikzpicture}[
    scale=0.9,
    font=\small,
    % Stile celle matrici
    cell/.style={
        rectangle, 
        draw=black!80, 
        minimum size=7mm, 
        inner sep=0pt, 
        anchor=center,
        outer sep=0pt,
        fill=white
    },
    % Colori desaturati coerenti con i tuoi esempi
    pi1_color/.style={fill=red!10},
    pi2_color/.style={fill=yellow!15},
    pi3_color/.style={fill=blue!10},
    input_color/.style={fill=gray!5},
    % Stili testo
    header_text/.style={font=\small, color=black!90},
    index_text/.style={font=\scriptsize\ttfamily, color=gray!80!black},
    % Evidenziatori (Senza libreria 'fit')
    highlight_red/.style={circle, draw=red!80!black, line width=1.2pt, minimum size=1.1em},
    highlight_orange/.style={circle, draw=orange!90!black, line width=1.2pt, minimum size=1.1em},
    arrow_red/.style={->, >=Stealth, red!80!black, line width=1pt},
    arrow_orange/.style={->, >=Stealth, orange!90!black, line width=1pt}
]

% --- 1. PERMUTATIONS PI_1, PI_2, PI_3 ---
\matrix (pi) [matrix of nodes, nodes={cell}, column sep=-\pgflinewidth, row sep=-\pgflinewidth] {
|[pi1_color]| 2 & |[pi2_color]| 4 & |[pi3_color]| 3 \\
|[pi1_color]| 3 & |[pi2_color]| 2 & |[pi3_color]| 4 \\
|[pi1_color]| 7 & |[pi2_color]| 1 & |[pi3_color]| 7 \\
|[pi1_color]| 6 & |[pi2_color]| 3 & |[pi3_color]| 2 \\
|[pi1_color]| 1 & |[pi2_color]| 6 & |[pi3_color]| 6 \\
|[pi1_color]| 5 & |[pi2_color]| 7 & |[pi3_color]| 1 \\
|[pi1_color]| 4 & |[pi2_color]| 5 & |[pi3_color]| 5 \\
};

% Etichette pi_i sopra le colonne
\node[above=2pt of pi-1-1.north] {$\pi_1$};
\node[above=2pt of pi-1-2.north] {$\pi_2$};
\node[above=2pt of pi-1-3.north] {$\pi_3$};

% --- 2. INPUT MATRIX ---
\matrix (input) [matrix of nodes, nodes={cell, input_color}, right=1.2cm of pi, column sep=-\pgflinewidth, row sep=-\pgflinewidth] {
1 & 0 & 1 & 0 \\
1 & 0 & 0 & 1 \\
0 & 1 & 0 & 1 \\
0 & 1 & 0 & 1 \\
0 & 1 & 0 & 1 \\
1 & 0 & 1 & 0 \\
1 & 0 & 1 & 0 \\
};

% Etichette Documenti sopra la matrice di input (D1..D4)
\foreach \i in {1,...,4} {
    \node[index_text, above=2pt of input-1-\i.north] {$D_\i$};
}

% --- 3. SIGNATURE MATRIX ---
\matrix (sig) [matrix of nodes, nodes={cell}, right=1.5cm of input, column sep=-\pgflinewidth, row sep=-\pgflinewidth] {
|[pi1_color]| 2 & |[pi1_color]| 1 & |[pi1_color]| 2 & |[pi1_color]| 1 \\
|[pi2_color]| 2 & |[pi2_color]| 1 & |[pi2_color]| 4 & |[pi2_color]| 1 \\
|[pi3_color]| 1 & |[pi3_color]| 2 & |[pi3_color]| 1 & |[pi3_color]| 2 \\
};

% Titolo richiesto sopra la Signature Matrix
\node[header_text, above=12pt of sig] {Matrice $SIG(\pi, C)$};

% Etichette Documenti sopra la Signature Matrix
\foreach \i in {1,...,4} {
    \node[index_text, above=2pt of sig-1-\i.north] {$D_\i$};
}

% --- FRECCE DI FLUSSO (opzionali ma nello stile dei tuoi esempi) ---
\draw[->, >=Stealth, color=gray!40, thick] ([xshift=2mm]input.east) -- ([xshift=-5mm]sig.west);


    % --- FRECCE E LOGICA (Layer superiore) ---

    % ESEMPIO 1 (Rosso): Calcolo di SIG(pi1,sud D1) = 2
    % Disegno i cerchi direttamente sulle coordinate dei nodi esistenti
    \node[highlight_red] (h_in_1) at (input-1-1) {};
    \node[highlight_red] (h_pi_1) at (pi-1-1) {};
    \node[highlight_red, rectangle, minimum size=7mm] (h_sig_1) at (sig-1-1) {};

    % Frecce rosse
    \draw[arrow_red] (h_in_1.west) to[out=200, in=-20] (h_pi_1.south); 
    \draw[arrow_red] (h_pi_1.north) to[out=45, in=150] node[midway, above, font=\scriptsize, sloped] {MinHash} (h_sig_1.north west);

    % ESEMPIO 2 (Arancione): Calcolo di SIG(pi2, D3) = 4
    % Disegno i cerchi direttamente sulle coordinate dei nodi esistenti
    \node[highlight_orange] (h_in_2) at (input-1-3) {};
    \node[highlight_orange] (h_pi_2) at (pi-1-2) {};
    \node[highlight_orange, rectangle, minimum size=7mm] (h_sig_2) at (sig-2-3) {};

    % Frecce arancioni
    \draw[arrow_orange] (h_in_2.west) to[out=200, in=-20] (h_pi_2.south);
    \draw[arrow_orange] (h_pi_2.north) to[out=60, in=130] (h_sig_2.north west);

    % --- LEGENDA ---
    \node[below=5mm of input, align=center, font=\footnotesize, color=gray!30!black] {
        Logica MinHashing: $\min \{ \pi(r) \mid \text{Input}[r,c] = 1 \}$\\
        \textcolor{red!80!black}{\textbf{Esempio 1}}: $D_1$ ha 1 in riga 1 $\rightarrow$ $\pi_1(1) = 2$. Firma = 2.\\
        \textcolor{orange!90!black}{\textbf{Esempio 2}}: $D_3$ ha 1 in riga 1 $\rightarrow$ $\pi_2(1) = 4$. Firma = 4.
    };

        \foreach \i in {1,...,7} {
            \node[index_text, left=1pt of pi-\i-1.west] {$\i$};
        }

        % ETICHETTE RIGA (pi_1, pi_2, pi_3) a sinistra di SIG
        \foreach \i in {1,...,3} {
            \node[index_text, left=0pt of sig-\i-1.west] {$\pi_\i$};
        }
    \end{tikzpicture}
    \caption{Processo di generazione della Signature Matrix tramite Minhashing.}
    \label{fig:3}
\end{figure}