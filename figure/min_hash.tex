\begin{figure}[ht]
\centering
\begin{tikzpicture}[
    scale=0.9,
    font=\small,
    % Stile celle matrici
    cell/.style={
        rectangle, 
        draw=black!80, 
        minimum size=7mm, 
        inner sep=0pt, 
        anchor=center,
        outer sep=0pt,
        fill=white
    },
    % Colori desaturati coerenti con i tuoi esempi
    pi1_color/.style={fill=red!10},
    pi2_color/.style={fill=yellow!15},
    pi3_color/.style={fill=blue!10},
    input_color/.style={fill=gray!5},
    % Stili testo
    header_text/.style={font=\small, color=black!90},
    index_text/.style={font=\scriptsize\ttfamily, color=gray!80!black}
]

% --- 1. PERMUTATIONS PI_1, PI_2, PI_3 ---
\matrix (pi) [matrix of nodes, nodes={cell}, column sep=-\pgflinewidth, row sep=-\pgflinewidth] {
|[pi1_color]| 2 & |[pi2_color]| 4 & |[pi3_color]| 3 \\
|[pi1_color]| 3 & |[pi2_color]| 2 & |[pi3_color]| 4 \\
|[pi1_color]| 7 & |[pi2_color]| 1 & |[pi3_color]| 7 \\
|[pi1_color]| 6 & |[pi2_color]| 3 & |[pi3_color]| 2 \\
|[pi1_color]| 1 & |[pi2_color]| 6 & |[pi3_color]| 6 \\
|[pi1_color]| 5 & |[pi2_color]| 7 & |[pi3_color]| 1 \\
|[pi1_color]| 4 & |[pi2_color]| 5 & |[pi3_color]| 5 \\
};

% Etichette pi_i sopra le colonne
\node[above=2pt of pi-1-1.north] {$\pi_1$};
\node[above=2pt of pi-1-2.north] {$\pi_2$};
\node[above=2pt of pi-1-3.north] {$\pi_3$};

% --- 2. INPUT MATRIX ---
\matrix (input) [matrix of nodes, nodes={cell, input_color}, right=1.2cm of pi, column sep=-\pgflinewidth, row sep=-\pgflinewidth] {
1 & 0 & 1 & 0 \\
1 & 0 & 0 & 1 \\
0 & 1 & 0 & 1 \\
0 & 1 & 0 & 1 \\
0 & 1 & 0 & 1 \\
1 & 0 & 1 & 0 \\
1 & 0 & 1 & 0 \\
};

% Etichette Documenti sopra la matrice di input (D1..D4)
\foreach \i in {1,...,4} {
    \node[index_text, above=2pt of input-1-\i.north] {$D_\i$};
}

% --- 3. SIGNATURE MATRIX ---
\matrix (sig) [matrix of nodes, nodes={cell}, right=1.2cm of input, column sep=-\pgflinewidth, row sep=-\pgflinewidth] {
|[pi1_color]| 2 & |[pi1_color]| 1 & |[pi1_color]| 2 & |[pi1_color]| 1 \\
|[pi2_color]| 2 & |[pi2_color]| 1 & |[pi2_color]| 4 & |[pi2_color]| 1 \\
|[pi3_color]| 1 & |[pi3_color]| 2 & |[pi3_color]| 1 & |[pi3_color]| 2 \\
};

% Titolo richiesto sopra la Signature Matrix
\node[header_text, above=12pt of sig] {Matrice $SIG(\pi, C)$};

% Etichette Documenti sopra la Signature Matrix
\foreach \i in {1,...,4} {
    \node[index_text, above=2pt of sig-1-\i.north] {$D_\i$};
}

% --- FRECCE DI FLUSSO (opzionali ma nello stile dei tuoi esempi) ---
\draw[->, >=Stealth, color=gray!40, thick] ([xshift=2mm]input.east) -- ([xshift=-2mm]sig.west);

\end{tikzpicture}
\caption{Processo di generazione della Signature Matrix tramite Minhashing.}
\end{figure}