
\subsubsection{Esempio 2}
Si mostra ora una famiglia  di funzioni hash \emph{2-indipendenti} sia $p \geq n$ un numero primo, siano $a,b \in \mathbb{Z}_p$ definite come nell'esempio precedente, si definisce la funzione hash $h_{a,b}$
\[
    h_{a,b}(x) = \left[ax + b \mod p\right] \mod m
\]
in pratica la funzione hash è definita come un polinomio scelto uniformemente in $\mathbb{Z}_p$ di grado 1. Si definisce la famiglia $\hat{\mathcal{H}}$
\[
    \hat{\mathcal{H}} = \{ h_{a,b} \mid a,b \in \mathbb{Z}_p\}
\] 
\begin{theorem}[2-indipendenza e universalità]
    $\hat{\mathcal{H}}$ è una famiglia \emph{2-indipendente} e universale
\end{theorem}
\begin{proof}
    Sia $X = (ax + b) \mod p$ e $Y = (ay +b) \mod p$ per $x \neq y$. Poiché $a \neq 0$ e $p > n$ allora $X \neq Y$.
    \[
        h_{a,b}(x) = \left[h_{a,b}(y) \iff X = Y \mod p\right] \mod m
    \]
    \begin{enumerate}
        \item $X$ e $Y$ sono distribuite in modo uniforme su $\mathbb{Z}_p$, poiché $a,b$ sono, a loro volta, distribuite in modo uniforme e $h$ è lineare iniettiva.
        \begin{proof}
            \begin{align}
                    ax + b \equiv_p ay + b
                &\iff
                    a(x-y) \equiv_p 0
                \\
                \implies (x-y) \equiv_p 0
                &\iff x = y
            \end{align}
        Il passaggio (6) è giustificato poichè $ x,y < p-1$
        \end{proof}
        
        \item $X$ e $Y$ sono quasi indipendenti tra loro, $\Pr[X = i \land Y = j] = \frac{1}{(p-1)p}$
        \begin{proof}
            \begin{equation}
                \begin{cases*}
                    ax + b \equiv_p i\\
                    ay + b \equiv_p j\\
                \end{cases*}
            \end{equation}
            Esiste un unica soluzione per $(a,b) \in {\mathbb{Z}_p}^2$, poiché il rango è 2.
            Siano $f,g$ soluzioni uniche del sistema lineare 
            \begin{align*}    
                \Pr\left[ (ax + b) \equiv_p i \land (ay + b) \equiv_p j\right] &= Pr[a = f(x,y,i,j) \land b = g(x,y,i,j) ]\\
                &= \Pr[a = f(x,y,i,j)] \cdot \Pr[b = g(x,y,i,j)]
            \end{align*}
            Per cui
            \begin{align*}    
                \Pr\left[ Y = j\mid X = i\right] &= \frac{\Pr[X = i \land Y = j]}{\Pr[X = i]}\\
                &= \frac{1}{p-1}
            \end{align*}    
        \end{proof}    
    \end{enumerate}
    Per una $i$ fissata in $\mathbb{Z}_p$ ci sono al massimo $\frac{p}{m}-1 \leq \frac{(p-1)}{m}$ valori per $Y$ tale che $Y = i \mod m$, ovvero tutti gli interi $\in \mathbb{Z}_p$ per cui la distanza da $i$ è un multiplo di $m$. Da 2 $\Pr[Y = j \mid X = i] = \frac{1}{p-1}$ per l'union buond si ottiene $\Pr[Y=i \mod m \mid X = i] \leq \frac{p-1}{m} frac{1}{p-1} = \frac{1}{m}$.
\end{proof}