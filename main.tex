\documentclass{article}
\usepackage[margin=2cm]{geometry}
\usepackage[utf8]{inputenc}
\usepackage[italian]{babel}
\usepackage{amsmath,amssymb}
\usepackage{amsthm}
\usepackage{hyperref}
\usepackage{empheq}
\usepackage{booktabs}
\usepackage{amsthm}
\usepackage{imakeidx}
\usepackage{tikz}
\usepackage[ruled,vlined]{algorithm2e}
\usetikzlibrary{arrows.meta, positioning, shapes.geometric, calc, decorations.pathreplacing, calligraphy}
\theoremstyle{remark}
\newtheorem{example}{Esempio}
\theoremstyle{remark}
\newtheorem{theorem}{Teorema}[section]
\newtheorem{lemma}{Lemma}[subsection]
\newtheorem{definition}{Definizione}[subsection]
\DeclareMathOperator{\Var}{Var}
\newcommand{\E}[1]{\mathbb{E}\!\left[#1\right]}

\begin{document}
    \begin{titlepage}
\vspace*{3em}
    \begin{center}
        \LARGE{Appunti di Big Data}
    \end{center}
\vspace*{3em}
\end{titlepage}
    \tableofcontents
    \newpage
    \section{Richiami di probabilità}
    Iniziamo introducendo i concetti fondamentali della teoria della probabilità, necessari per analizzare rigorosamente le prestazioni e la correttezza degli algoritmi probabilistici che verranno presentati nelle sezioni successive.

\subsection{Spazi di campionamento e funzioni di probabilità}

Il punto di partenza per ogni analisi probabilistica è la definizione dell'ambiente in cui operiamo.

\begin{definition}[Spazio di campionamento]
    Uno \emph{spazio di campionamento} $\Omega$ è l'insieme di tutti i possibili esiti di un esperimento aleatorio. I sottoinsiemi di $\Omega$ prendono il nome di \emph{eventi}. Un evento $A$ si dice verificato se l'esito dell'esperimento è un elemento di $A$.
\end{definition}

Per quantificare la probabilità che un evento si verifichi, associamo allo spazio di campionamento una misura
\begin{definition}[Funzione di probabilità]
    Una funzione di probabilità è una funzione $\Pr: \mathcal{F} \to [0,1]$ (dove $\mathcal{F}$ è la famiglia degli eventi) che soddisfa i seguenti assiomi:
    \begin{itemize}
        \item \textbf{Normalizzazione:} $\Pr{[\Omega]} = 1$.
        \item \textbf{Additività numerabile:} Per ogni sequenza di eventi a due a due disgiunti $E_1, E_2, \dots$ si ha:
        \[
            \Pr{\left[\bigcup_{i=1}^{\infty}E_i\right]} = \sum_{i=1}^{\infty}\Pr{[E_i]}
        \]
    \end{itemize}
\end{definition}

Da questi assiomi fondamentali derivano diverse proprietà utili per il calcolo probabilistico, tra cui la probabilità del complemento $\Pr[\bar{E}] = 1 - \Pr[E]$ e il principio di inclusione-esclusione per eventi non disgiunti:
\[
    \Pr{[E_1 \cup E_2]} = \Pr{[E_1]} + \Pr{[E_2]} - \Pr{[E_1 \cap E_2]}
\]

Un risultato noto è lo \emph{Union-bound}, che permette di limitare la probabilità che si verifichi almeno uno tra diversi eventi, senza conoscerne la dipendenza
\[
    \Pr{\left[\bigcup_{i=1}^{n}E_i\right]} \leq \sum_{i=1}^{n}\Pr{[E_i]}
\]

\subsection{Probabilità condizionata e Teoremi fondamentali}

Spesso l'esito di un esperimento è influenzato da informazioni parziali già disponibili. Questo concetto è formalizzato dalla probabilità condizionata.

\begin{definition}[Probabilità condizionata]
    La probabilità condizionata di un evento $E_1$ dato il verificarsi di un evento $E_2$ (con $\Pr[E_2] > 0$) è definita come:
    \[
        \Pr{[E_1|E_2]} = \frac{\Pr{[E_1 \cap E_2]}}{\Pr{[E_2]}}
    \]
\end{definition}

L'introduzione della probabilità condizionata ci permette di definire il concetto di \textbf{indipendenza}: due eventi $E_1, E_2$ sono indipendenti se la conoscenza del verificarsi di uno non altera la probabilità dell'altro, ovvero $\Pr[E_1|E_2] = \Pr[E_1]$, che equivale a richiedere $\Pr[E_1 \cap E_2] = \Pr[E_1] \cdot \Pr[E_2]$.

Per analizzare eventi complessi che dipendono da diverse `'cause'' disgiunte, utilizziamo il Teorema delle Probabilità Totali e il Teorema di Bayes.

\begin{theorem}[Teorema delle probabilità totali]
    Sia $E_1, E_2, \dots, E_n$ una partizione dello spazio campionario $\Omega$ (ovvero eventi mutualmente disgiunti $\bigcup_{i=1}^{n} E_i = \Omega$.). Per ogni evento $E \subseteq \Omega$ vale:
    \[
        \Pr{[E]} = \sum_{i=1}^{n}\Pr{[E|E_i]}\cdot\Pr{[E_i]}
    \]
\end{theorem}

\begin{theorem}[Teorema di Bayes]
    Siano $E_1,E_2,\dots,E_n$ eventi disgiunti tali per cui $\bigcup_{i=1}^{n} E_i = \Omega$, la probabilità condizionata di $E_i$ dato un evento arbitrario $E$ può essere espressa come:
    \[
    \Pr[E_i|E] = \frac{\Pr[E|E_i]\Pr[E_i]}{\sum_{j=1}^n \Pr[E|E_j]\Pr[E_j]} = \frac{\Pr\left[E|E_i\right]\Pr\left[E_i\right]}{\Pr\left[E\right]}
    \]
\end{theorem}

\subsection{Variabili aleatorie}

In molte applicazioni non siamo interessati all'esito elementare dell'esperimento, ma a un valore numerico ad esso associato (ad esempio, il tempo di esecuzione di un algoritmo).

\begin{definition}[Variabile aleatoria discreta]
    Una variabile aleatoria $X$ è una funzione $X: \Omega \to \mathbb{R}$. Essa si dice discreta se la sua immagine è un insieme finito o infinito numerabile.
\end{definition}

\begin{definition}[Valore atteso]
    Il valore atteso di una variabile aleatoria discreta $X$, denotato con $\E{X}$, è definito come:
    \[
        \E{X} = \sum_{i=1}^{n}i\Pr\left[X=x\right]
    \]
\end{definition}

Una proprietà fondamentale per lo studio degli algoritmi probabilistici è la \textbf{linearità del valore atteso}, la quale afferma che il valore atteso della somma di più variabili aleatorie è pari alla somma dei loro valori attesi, \emph{indipendentemente} dal fatto che esse siano indipendenti o meno:
\[
    \E{\sum_{i=1}^{n}X_i} = \sum_{i=1}^{n}\E{X_i}
\]

Infine, estendiamo il concetto di indipendenza alle variabili aleatorie. Un insieme di variabili $X_1, \dots, X_n$ è detto \emph{$k$-wise indipendente} se ogni sottoinsieme di $k$ variabili soddisfa la condizione di indipendenza numerica. Se questa condizione vale per $k=n$, le variabili si dicono completamente indipendenti.

Anche il valore atteso può essere condizionato a un particolare evento, sia $X$ una variabile aleatorie e $\mathcal{E}$ un evento, il valore atteso di $X$ condizionato $\mathcal{E}$ è definito come 
\[
    \E{X|\mathcal{E}} = \sum_{x}x\Pr{\left[X = x|\mathcal{E}\right]}
\]
con $x$ valori assumibili da $X$.

La \emph{varianza} misura quanto il valore di una variabile aleatoria si discosta globalmente dal valore atteso.


\begin{definition}[Varianza]
    La varianza su $X$ è definita come
    \[
    \Var\left[X\right] = \E{\left(X-\E{X}^2\right)} = \E{X^2} - \E{X}^2
    \]
\end{definition}
La varianza di una variabile aleatoria è calcolata sommando il prodotto del quadrato della differenza tra il valore assunto e il valore atteso, con la probabilità che tale variabile assuma quei valori.
Sia $\mu = \E{X}$ e $X$ una variabile aleatoria che assume i valori $\{1,\dots,n\}$
\[
    \Var[X] = \sum_{k=1}^{n}(k-\mu)^2\Pr{[X=k]}
\]
La deviazione standard invece si definisce come segue
    \[
        \sigma[X] = \sqrt{\Var[X]}
    \]

\subsection{Distribuzioni di probabilità}
Una distribuzione di probabilità è un modello matematico che associa una probabilità a ogni possibile risultato di una variabile aleatoria. Ne esistono diversi tipi, tra cui si presentano la \emph{Distribuzione binomiale} e la \emph{Distribuzione geometrica}.

\begin{definition}[Distribuzione binomiale]
    Si supponga di eseguire un esperimento la cui probabilità di successo è $p \in [0,1]$, la probabilità di fallimento sarà dunque $1-p$. L'esperimento è descritto da una variabile aleatoria $Y$ così definita
    \[
        Y = \begin{cases*}
            1 \text{ se l'esperimento ha successo}\\
            0 \text{ altrimenti}
        \end{cases*}
    \]
    $Y$ è una variabile aleatoria binaria che assume valori \{0,1\}, $Y$ è chiamata variabile bernoulliana.
\end{definition}
\emph{Osservazione:}\[
    \E{Y} = p\cdot1 + (1-p)\cdot 0 = p = \Pr[Y=1]
\]

Una sequenza di $n$ esperimenti indipendenti, ogniuno con probabilità di successo $p$, è chiamata \emph{processo di Bernoulli}.

Sia $X$ la variabile aleatoria che conta il numero di successo negli $n$ esperimenti con probabilità di successo $p$, $X$ ha distribuzione binomiale, denotata $X\sim Bin(n,p)$ ed è definita dalla seguente distribuzione di probabilità
\[
    \forall j = 0,\dots,n \quad \Pr\left[X = j\right] = \binom{n}{j}p^{j}\left(1-p\right)^{n-j}
\]

il fattore $p^j$ rappresenta i $j$ successi, al contrario $(1-p)^{n-j}$ rappresenta gli $n-j$ fallimenti.

Passando ora al calcolo del valore atteso di $X$ (variabile aleatoria con distribuizione binomiale)
\begin{align*}    
    \E{X} &= \sum_{j=0}^{n}j\binom{n}{j}p^j\left(1-p\right)^{n-j}\\
    &= \sum_{j=0}^{n}j \frac{n!}{\left(n-j\right)! j!}p^j\left(1-p\right)^{n-j} = \sum_{j=1}^{n}\frac{\left(n-1\right)!}{\left(j-1\right)!\left(n-j\right)!}p^j\left(1-p\right)^{n-j}\\
    &= np\sum_{j=1}^{n}\frac{\left(n-1\right)!}{\left(j-1\right)!\left(\left(n-1\right)-\left(j-1\right)\right)!}p^{j-1}\left(1-p\right)^{((n-1)-(j-1))}\\
    \substack{(k = j-1)} &= np\sum_{k=0}^{n-1}\frac{\left(n-1\right)!}{k!\left(\left(n-1\right)-k\right)!}p^{k}\left(1-p\right)^{(n-1)-k}\\
    &= np \sum_{k=0}^{n-1} \binom{n-1}{k}p^k(1-p)^{(n-1)-k} = np
\end{align*}

Si fa presente, come, sfruttando la linearità del valore atteso, quest'ultimo può essere calcolato così: 

se $X\sim Bin(n,p)$ allora $X=\#Successi$ su $n$ prove ogniunga delle quali con probabilità di successo $p$ $Y_i$ è la variabile aleatoria che tiene conto del successo per ogni esperimento \emph{i-esimo} come da un risultato precedente $\E{Y_i} = \Pr{[Y_i = 1]} = p$
\[
    X = \sum_{i = 1}^nY_i \implies \E{X} = \E{\sum_{i = 1}^n Y_i} = \sum_{i = 1}^n\E{Y_i} = np
\] 

Similmente si procede con il calcolo della varianza di $X$ (variabile binomiale)
\begin{align*}
    \E{X^2} &= \sum_{j=0}^{n}\binom{n}{j}{p^j(1-p)^{n-j}}j^2\\
    &= \sum_{j=0}^{n}\frac{n!}{\left(n-j\right)!}{\left((j-j^2)+j\right)}\\
    &= \sum_{j=0}^{n} \frac{n!(j-j^2)}{\left(n-j\right)!}{p^j(1-p)^{n-j}} + \sum_{j=0}^{n} \frac{n!j}{\left(n-j\right)!}{p^j(1-p)^{n-j}} \\
    &= np \sum_{j=0}^{n}\frac{\left(n-1\right)}{\left(n-j\right)!\left(j-1\right)!}{p^{j-1}(1-p)^{n-j}}\\
    &= n(n-1)p^2 + np\\
    \Var[X] &=\E{X^2}-\left(\E{X}\right)^2 = n(n-1)p^2+np -n^2p^2 = \boxed{np(1-p)}
\end{align*}
In modo alternativo possiamo valutare la varianza delle variabili $Y_i$  
\begin{align*}
    \Var[X] &= \Var\left[\sum_{i=1}^n{Y_i}\right] = \sum_{i=1}^n\E{(Y_i-\E{Y_i}^2)}\\
    &= n(p(1-p)^2 + (1-p)(-p)^2) = \boxed{np(1-p)} 
\end{align*}
    \section{Fare uso di algoritmi probabilistici}
    È naturale domandarsi riguardo la scelta di utilizzare algoritmi \emph{probabilistici} per cui c'è una possibilità di ottienere un risultato errato, rispetto agli algoritmi \emph{deterministici} in cui si ha la certezza di ottenere sempre un risultato corretto. In questa sezione si vedono degli algoritmi di esempio che giustificano questa scelta.
    \subsection{Verifica Identità polinomiali}
    \subsubsection{Algoritmo deterministico}
        Iniziamo descrivendo brevemente il problema, siano $f(x)$ e $g(x)$ due polinomi per cui $d = \deg(f) = \deg(g)$ per cui
        \[
            f(x) = \prod^d_{i=0}{(x-a_i)}
        \]
        mentre $g(x)$ è descritto nel modo seguente
        \[
            g(x) = \sum^{d}_{i=0}{(c_i \cdot x^i)}
        \]
        verificare l'identità $\forall x \ f(x) = g(x) $.
        
        Un'opzione è quella di convertire $f(x)$ nella forma canonica moltiplicando i fattori tra di loro, così facendo è sufficiente verificare che i coefficienti di ciascuna variabile siano uguali per i due polinomi ovvero per $a_i$ coefficienti di $f(x)$ e $c_i$ i coefficienti di $g(x)$ associati allo stesso $x^i$.
        $$ f(x) = g(x) \iff \forall i = 0,\dots,d \ a_i = c_i$$
        Questa soluzione è \textit{deterministica} e richiede $O(d^2)$ operazioni
        
    \subsubsection{Algoritmo probabilistico}
    Sia $d$ il grado dei due polinomi, come in precedenza. l'algoritmo sceglie u.a.r. un intero $r \in_u \{0,\dots,100d\}$ e calcola i valori di $f(r)$ e $g(r)$
    \begin{equation*}
        Output = 
        \begin{cases*}
            Vero \text{ se } f(r) = g(r)\\
            Falso \text{ altrimenti}
        \end{cases*}
    \end{equation*}
    Assumendo che la scelta di $r \in_u \{1,  \dots, 100d\}$ sia $O(1)$ e il calcolo di $f(r)$ e $g(r)$ sia $O(d)$, la complessità dell'algoritmo è $O(d)$.

    L'algoritmo probabilistico è molto migliore rispetto alla sua controparte deterministica, ma questo miglioramento comporta un costo non indifferente, l'algoritmo probabilistico in qualche caso potrebbe sbagliare.

    \vspace{2em}
    \emph{In quali casi l'algoritmo sbaglia?}
    \begin{example}
        \[
            F(x) = (x+2)(x+2)
            \quad
            G(x) = x^2+7x+1
        \]
        I due polinomi sono ovviamente diversi,
        per $r = 2$ si ha $F(2) = 19$ e $G(2) = 16$, mentre per $r=1$ $F(1) = 9 = G(1)$ 
        ne segue che se l'algoritmo secgliesse $r = 2$ si arriverebbe a una risposta errata.
    \end{example}
    Sia $f(x) \neq g(x)$, e la somma dei gradi di $x$ in $f$ e $g$ è limitato da $d$. Definiamo $h(x)$ come segue:
    \[
        h(x):= f(x) - g(x)
    \]
    naturalmente anche $h(x)$ è limitato nel grado da $d$, e l'equazione $h(x) = 0$ ha al massimo $d$ soluzioni. È banale convincersi che il caso $h(x) = 0$ descrive esattamente i casi in cui l'algoritmo probabilistico sbaglia.

    \subsubsection*{Analisi dell'algoritmo}
    \begin{enumerate}
        \item Se l'identità $f(x) \equiv g(x)$ è vera, l'algoritmo da \emph{sempre} in output la risposta corretta
        \item Se l'identità $f(x) \equiv g(x)$ non è vera, l'algoritmo sbaglia in \emph{alcuni} casi
    \end{enumerate}
    Si ha che scegliendo $r \in_u [1,100d]$ allora \fbox{$\Pr[f(r) = g(r)] < \frac{1}{100}$} \label{ris:1}(la probabilità che l'algoritmo sbagli è bassa).
    
    \vspace{1em}
    Si generalizza l'algoritmo per $k$ prove indipendenti
    \begin{itemize}
        \item Evento semplice $E_s$: fissata una sequenza specifica $r_1,\dots,r_k \in_u [1,100d]$, l'evento semplice rappresenta la scelta dell'algoritmo di quella specifica sequenza
        \item  Evento negativo $E_n$: la sequenza scelta contiene tutte radici di $h(x)$
        \item Se in qualsiasi delle $k$ prove l'output è $Falso$ restituisci $Falso$
    \end{itemize}
    Si calcolano la probabilità di $E_s$ ed $E_n$
    \[
        \Pr{[E_s]} = \left({\frac{1}{100d}}\right)^k
        \quad
        \Pr{[E_n]} \leq d^k \cdot \left( \frac{1}{100d} \right)^k = { \left( \frac{1}{100} \right) }^k
    \]
    Ci sono al massimo $d^k$ sequenze in cui tutte le $r_i$ scelte sono radici di $h(x)$, inoltre da un risultato precedente (\ref{ris:1}), la probabilità che una sequenza contenga una radice è $\leq \frac{1}{100}$. Poiché le scelte sono indipendenti, la probabilità di $k$ scelte successive errate è proprio ${ \left( \frac{1}{100} \right) }^k$.
    
    \vspace{1em}
    \noindent
    L'evento $E_n$ rappresenta quindi l'evento in cui l'algoritmo da un risultato errato, questo avviene con probabilità $\leq { \left( \frac{1}{100} \right) }^k$.

    \subsection{Verifica Moltiplicazione matriciale}
    Siano $A,B,C \in \{0,1\}^{n\times n}$ tre matrici binarie, verificare 
    \[
        A\cdot B = C
    \]
    \subsubsection{Algoritmo deterministico}
    L'algoritmo \emph{deterministico} richiede l'uso della moltiplicazione tra matrici,si ricorda la moltiplicazione tra matrici, sia $AB = A \cdot B$
    \[
        AB = 
        \begin{bmatrix}
            c_{11} & \cdots & c_{1n} \\ 
            \vdots & \ddots & \vdots \\
            c_{n1} & \cdots & c_{nn}
        \end{bmatrix}
    \]
    In cui ogni termine $c_{ij} = \sum_{k=1}^{n}a_{ik}b_{kj}$. Un algoritmo banale impiega $\Theta(n^3)$ tempo con un metodo standard, o $\Theta(n^{2.37})$ con un metodo più efficiente. 
    
    \subsubsection{Algoritmo probabilistico}
    Iniziamo subito definendo il comportamento di un algoritmo randomizzato, nettamente più efficiente dell'algoritmo deterministico visto in precedenza
    \begin{enumerate}
        \item Scelta di un vettore $\bar{r} = (r_1,\dots,r_n) \in_u \{0,1\}^n$
        \item Calcolo di $B\cdot\bar{r}$
        \item Calcolo $A(B\cdot \bar{r})$
        \item Calcolo $C\cdot\bar{r}$
    \end{enumerate}
    \[
    Output = 
    \begin{cases*}
        Vero \text{ se } A(B\cdot \bar{r}) = C\cdot\bar{r}\\
        Falso \text{ altrimenti }    
    \end{cases*}
    \]
    L'algorimo esegue moltiplicazioni tra le matrici e il vettore scelto $\bar{r}$, più efficienti rispetto alla moltiplicazione tra matrici, di seguito un analisi per determinare l'efficacia in termini probabilistici dell'algoritmo.

    \subsubsection*{Analisi}
    Il seguente risultato fornisce una base utile per determinare la probabilità di successo dell'algoritmo
    \begin{theorem}
        Se $AB \neq C$ e $r$ è scelto uniformemente da $\{0,1\}$ allora 
        \[
            \Pr{[AB\cdot\bar{r} = C\cdot\bar{r}]} \leq \frac{1}{2}
        \]
    \end{theorem}
    Si ricorda che scegliere $\bar{r} \in_u \{0,1\}^n$ è equivalente a scegliere $n$ valori indipendenti $r_1,\dots,r_n \in_u \{0,1\}$  
    \begin{proof}
        Sia $D=AB - C \neq 0$ una matrice in $\{0,1\}^{n\times n}$ diversa da 0, ovvero $\exists d_{ij} \neq 0$
        \[
            AB\cdot\bar{r} = C\cdot\bar{r} \implies D\cdot\bar{r} = 0
        \]
        Per semplicità assumiamo che $d_{ij} \neq 0$ sia proprio $d_{11}$. Poiché $D\cdot\bar{r} = 0$ 
        \[
            \sum_{j=1}^nd_{1j}r_j = 0 \quad \text{ ovvero } \quad r_1 = -\frac{\sum_{j=2}^n{d_{1j}r_j}}{d_{11}}
        \]
        Assumiamo di aver fissato una sequenza $r_2,\dots,r_n$ allora è possibile determinare facilmente $r_1$. La probabilità che $r_1$ rispetti l'uguaglianza è $\leq \frac{1}{2}$

        \vspace{1em}

        Sia $X = (x_2,\dots,x_n) \in \{0,1\}^{n-1} = B^{n-1}$
        \begin{align*}
            \Pr{[AB\cdot\bar{r} = C\cdot\bar{r}]} &= \sum_{ x \in B^{n-1}}\Pr{[AB\cdot\bar{r} = C\cdot\bar{r} \mid(r_2,\dots,r_n) = (x_2,\dots,x_n)]} \cdot \Pr{[(r_2,\dots,r_n) = (x_2,\dots,x_n)]} =\\
            &= \sum_{x \in B^{n-1}}\Pr{[ (AB\cdot\bar{r} = C\cdot\bar{r}) \cap (r_2,\dots,r_n) = (x_2,\dots,x_n)]}\\
            &\leq \sum_{x \in B^{n-1}}\Pr{\left[\left(r_1 = -\frac{\sum_{j=2}^n{d_{1j}r_j}}{d_{11}}\right) \cap (r_2,\dots,r_n) = (x_2,\dots,x_n)\right]} =\\
            &= \sum_{x \in B^{n-1}}\Pr{\left[r_1 = -\frac{\sum_{j=2}^n{d_{1j}r_j}}{d_{11}}\right]} \cdot \Pr{[(r_2,\dots,r_n) = (x_2,\dots,x_n)]}\\
            &\leq \sum_{x \in B^{n-1}}\frac{1}{2}\Pr{[(r_2,\dots,r_n) = (x_2,\dots,x_n)]}\\
            &= \frac{1}{2}.
        \end{align*}
    \end{proof}
    Anche in questo caso è possibile ripetere l'algoritmo su una sequenza di $k$ prove, si ha interesse a capire come cambia la confidenza sull'esito. Non avendo informazioni sulla provenienza di $A,B,C$ è ragionevole assumere che $A\cdot B = C$ sia vero con probabilità $\frac{1}{2}$. A questo scopo si definiscono gli eventi $E$ che definisce la correttezza dell'identità, $B$ l'evento in cui l'algoritmo restituisce $Vero$.
    
    \vspace{1em}
    Si inzia con $\Pr(E_0) = \Pr(\bar{E_0}) = \frac{1}{2}$ e poiché l'algorimo è a \emph{one-sided-error} $\Pr(B\mid E) = 1$, e $\Pr(B \mid \bar{E}) \leq \frac{1}{2}$. Ovvero è banale convincersi che la probabilità di successo dell'algorimo, data un'identità falsa, è certo. Applicando Bayes si può ottenere una formula per determinare la confidenza della correttezza alla prova $i+1$ sapendo la confidenza e il risultato della prova \textit{i-esima}
    \[
        \Pr{[E_{i+1}]} =  \frac{\Pr(B_i\mid E_i)\Pr{[E_i]}}{\Pr(B_i\mid E_i)\Pr{[E_i]} + \Pr(B_i\mid \bar{E_i})\Pr{[\bar{E_i}]}}
    \]
    In generale se prima di eseguire l'algoritmo l'\textit{i-esima} volta $\Pr{[E_i]} \geq \frac{2^i}{(2^i+1)}$ e l'algorimo restituisce $Vero$ ($\Pr{[B_i]}=1$) allora
    % TODO: Mostrare meglio i passaggi magari con un ref ad un'altra pagina
    \[
        \Pr{[E_{i+1} \mid B_i]} \geq \frac{\frac{2^i}{(2^i+1)}}{\frac{2^i}{(2^i+1)} + \frac{1}{2}\frac{1}{(2^i+1)}} = \frac{2^{i+1}}{2^{i+1}+1} = 1 - \frac{1}{2^i+1}
    \]
    La confidenza alla \textit{i-esima} prova è $\geq 1 - \frac{1}{2^i+1}$. Con questo risultato si può affermare che la confidenza nella corretta del risultato dell'algorimo cresce esponenzialmente nel numero di prove effettuate.
    
    \subsection{Problema Min-Cut}
    Spostiamo l'attenzione ora su un problema noto, il problema del \emph{Min-Cut}. Sia $G = \langle V,E \rangle$ un grafo. Un taglio (cut) è una partizione dei vertici $V$ in due sottoinsiemi non vuoti $A,B \subset V$ tali che $B = V-A$. L'insieme degli archi del taglio è definito come
    \[
        C = \{(u,v) \in E \mid u \in A, v \in B\}
    \]
    Il problema del \emph{min-cut} consiste nel trovare $C$ per cui $|C|$ è minima. $C$ inoltre è il minimo insieme di archi per cui il grafo $G$ è sconnesso.

    \subsubsection{Algoritmo probabilistico}
    L'algoritmo probabilistico per risolvere il Min-Cut è basato sull'operazione di \emph{contrazione} dei nodi. Il processo di contrazione avviene nel seguente modo
    \begin{enumerate}
        \item Si seleziona un arco $e=(u,v) \in_u E$ (in modo uniforme)
        \item Si fondono i nodi $u$ e $v$ in un unico nodo.
        \item Si rimuovono i \emph{self-loops} risultanti dalla contrazione.
    \end{enumerate}
    \begin{figure}[h]
    \centering
    \begin{tikzpicture}[
        node/.style={circle, draw=black, fill=gray!10, minimum size=7mm, thick},
        selected/.style={circle, draw=red, fill=red!10, minimum size=7mm, thick},
        contracted/.style={ellipse, draw=red, fill=red!20, minimum size=10mm, thick, font=\bfseries},
        edge/.style={thick},
        select_edge/.style={line width=2pt, red}
    ]

        % --- GRAFO ORIGINALE G ---
        \node[selected] (u) at (0,1) {$u$};
        \node[selected] (v) at (2,1) {$v$};
        \node[node] (a) at (-1,-0.5) {$a$};
        \node[node] (b) at (1,-1) {$b$};
        \node[node] (c) at (3,-0.5) {$c$};

        % Archi G
        \draw[select_edge] (u) -- (v) node[midway, above] {$e$};
        \draw[edge] (u) -- (a);
        \draw[edge] (u) -- (b); % Arco 1 verso b
        \draw[edge] (v) -- (b); % Arco 2 verso b
        \draw[edge] (v) -- (c);
        
        \node at (1, -1.8) {\textbf{Grafo $G$}};

        \draw[->, line width=1mm, -{Stealth[length=5mm]}] (3.5,0) -- (5.5,0);

        \begin{scope}[xshift=8cm]
            \node[contracted] (uv) at (1,1) {$uv$};
            \node[node] (a2) at (-1,-0.5) {$a$};
            \node[node] (b2) at (1,-1) {$b$};
            \node[node] (c2) at (3,-0.5) {$c$};

            % Archi G' con MULTI-ARCHI
            \draw[edge] (uv) -- (a2);
            % Rappresentazione di archi paralleli verso b
            \draw[edge] (uv) to[bend left=15] (b2); 
            \draw[edge] (uv) to[bend right=15] (b2);
            \draw[edge] (uv) -- (c2);
            
            % Self-loop rimosso
            \draw[red, dashed, thick] (uv) .. controls +(135:1.5cm) and +(45:1.5cm) .. (uv);
            \node[red] at (1, 2.2) {\small Self-loop rimosso};

            \node at (1, -1.8) {\textbf{Multigrafo $G'$ (Archi paralleli preservati)}};
        \end{scope}

    \end{tikzpicture}
    \caption{Rappresentazione della contrazione di un arco}
\end{figure}
    Si descrive ora l'algorimo probabilistico 
    \begin{enumerate}
        \item Ripeti $n-2$ volte:
            \begin{enumerate}
                \item Prendi $(u,v) \in_u E$
                \item Contrai $u,v$, ed elimina i \emph{self-loops}
            \end{enumerate}
        \item Restituisci l'insieme di archi che connettono i due vertici rimanenti
    \end{enumerate}

    \begin{theorem}
        L'algoritmo restituisce un \emph{min-cut} con probabilità $\geq \frac{1}{n(n-1)}$
    \end{theorem}

    \begin{lemma}
        La contrazione dei vertici non riduce la grandezza del \emph{min-cut}, può solo aumentare
    \end{lemma}
    \begin{proof}
        Ogni \emph{cutset} nel nuovo multigrafo è un \emph{cutset} nel grafo prima della contrazione
    \end{proof}

    \subsubsection*{Analisi dell'algorimo}
    Per analizzare la probabilità di successo dell'algorimo, si parte supponendo che il grafo abbia un min-cut $C$ di $k$ archi, si calcola ora la probabilità di trovare tale \emph{cutset} $C$.

    \begin{lemma}
        Se l'arco contratto non appartiene al min-cut $C$, allora nessun altro arco eliminato appartiene a $C$.
    \end{lemma}
    \begin{proof}
        Il processo di contrazione di $u$ e $v$ elimina gli archi paralleli $e_1 (u,v) \in E$, $e_2 (u,v) \in E$ ovvero tutti gli archi con estremi $u,v$. Tali archi appartengono o meno, contemporaneamente a $C$.
    \end{proof} 
    Si definiscono quindi due eventi $E_i$ che descrive l'evento in cui l'arco contratto all'iterazione \emph{i-esima} non appartiene a $C$, $F_i$ l'evento in cui nessun arco di $C$ è stato contratto nelle prime $i$ iterazioni $F_i = \bigcap_{j=1}^iE_j$. È facile vedere che $F_{n-2}$ è proprio l'evento in cui l'algorimo restituisce l'insieme corretto.

    \vspace{1em}\noindent
    Si parte notando che se $|C| = k$ allora la cardinalità di tutti i nodi è $\geq k$, inoltre il grafo ha almeno $\frac{nk}{2}$ archi. Per cui selezionando un arco $e \in E$ casualmente (in modo uniforme), la probabilità che $e$ sia un arco di $C$ è $\Pr(e\in E) \leq \frac{k}{\frac{nk}{2}} = \frac{2}{n}$, questo caso è proprio l'evento complementare di $E_1$, per cui $\Pr(E_1) \geq 1-\frac{2}{n}$. Ora assumendo che $E_1$ si sia verificato, rimangono $n-1$ nodi con un \emph{min-cut} di dimensione e grado minimo $\geq k$. Supponendo che in $i-1$ iterazioni l'algorimo non abbia mai selezionato nessun arco in $C$ ovvero verificando l'evento $F_{i-1}$ si deriva la formula
    \[
        \Pr{[E_i \mid F_{i-1}]} \geq 1 - \frac{k}{\frac{k(n-i+1)}{2}} = 1 - \frac{2}{n-i+1}
    \]
    
    Per determinare la probabilità di successo dell'algorimo si deve calcolare $\Pr{[F_{n-2}]}$.
    \begin{align*}
        \Pr{[F_{n-2}]} &= \Pr{[E_{n-2}\cap F_{n-3}]} = \Pr{[E_{n-2} \mid F_{n-3}]}\Pr{[F_{n-3}]} = \\
        &= \Pr{[E_{n-3} \mid F_{n-4}]}\cdots\Pr{[E_2\mid F_1]}\Pr{[F_1]} \geq\\
        &\geq \prod_{i=1}^{n-2}\left( 1 - \frac{2}{n-i+1} \right) = \prod_{i=1}^{n-2} \frac{n-1-1}{n-i+1} =\\
        &= \left( \frac{n-2}{n}\right) \left( \frac{n-3}{n-1}\right)\left( \frac{n-4}{n-2} \right) \cdots \left( \frac{2}{4}\right)\left(\frac{1}{3}\right) = \frac{2}{n(n-1)}
    \end{align*}
    Come al solito si analizza l'aumento della confidenza all'aumentare del numero di esecuzioni dell'algoritmo. Le diverse esecuzioni sono indipendenti per cui per $k$ prove si ha la probabilità di non trovare un \emph{min-cut} $ \leq \left( 1- \frac{2}{n(n-1)}\right)^k$. Per $k = n(n-1)\log n$ 
    si ha $ \left( 1- \frac{2}{n(n-1)}\right)^k \leq e^{-2\log n} = \frac{1}{n^2}$.
    \subsection{Quick Sort}
    Si passa ora al caso di un'altro algoritmo noto, il \emph{QuickSort}, un algoritmo di ordinamento che dato $S$ in input restituisce $S$ ordinato in output.
    \subsubsection{Algoritmo deterministico}
    Si descrivono i passaggi dell'algoritmo \emph{QuickSort} deterministico
    \begin{enumerate}
        \item Prendi il perno $y$ il primo elemento in $S$
        \item Confronta tutti gli elementi di $S$ con $y$, a tale scopo si definiscono due insiemi \[
            S_1 = \{ x \in S - \{y\} \mid x \leq y\} \quad S_2 = \{ x \in S-\{y\} \mid x > y\}
        \]
        \item Restituisci la lista $Det\_QS(S_1),y,Det\_QS(S_2)$
    \end{enumerate}
    \subsubsection{Algoritmo probabilistico}
    Qui di seguito sono riportati brevemente i passaggi dell'algoritmo
    \begin{enumerate}
        \item Scegli un elemento per il perno $y \in_u S$
        \item Confronta tutti gli elementi di $S$ con $y$, a tale scopo si definiscono due insiemi \[
            S_1 = \{ x \in S - \{y\} \mid x \leq y\} \quad S_2 = \{ x \in S-\{y\} \mid x > y\}
        \]
        \item Restituisci la lista $QS(S_1),y,QS(S_2)$
    \end{enumerate}
    \subsubsection*{Analisi degli algoritmi}
    In questo caso iniziamo analizzando l'algoritmo probabilistico. Siano $s_1,\dots,s_n$ gli elementi in $S$ ordinati, per determinare il numero di confronti eseguiti si definisce la variabile aleatoria binaria $X_{ij}$
    \[
        X_{ij} = \begin{cases*}
            1 \text{ se } s_i \text{ è confrontato con } s_j\\
            0 \text{ altrimenti }
        \end{cases*}
    \]
    Sia allora $T$ la variabile che conta il numero di confronti totali durante l'esecuzione dell'algoritmo
    \[
        T = \sum_{i=0}^n\sum_{j>i}X_{ij}
    \]
    poiché si sta considerando di un algorimo probabilistico si è interessati a calcolare il valore atteso di $T$, $\E{T}$. Si vuole calcolare $\Pr[X_{ij} = 1]$, $s_i$ è confrontato con $s_j$ solo nel caso in cui uno dei due elementi è scelto come \emph{perno} prima che gli $j-i-1$ elementi tra $s_i$ e $s_j$ siano scelti. Il perno è scelto in modo uniforme, per cui anche gli elementi $[s_i,\dots,s_j]$ sono scelti in modo uniforme, si ottienere
    \begin{align*}
        &\Pr{[X_{ij} = 1]} = \frac{2}{j-i+1}\\
        &\E{X_{ij}} = \frac{2}{j-i+1}
    \end{align*}
    Facendo uso delle proprietà del valore atteso
    \begin{align*}
        \E{T} &= \E{\sum_{i=0}^n\sum_{j>i}X_{ij}} = \sum_{i=0}^n\sum_{j>i}\E{X_{ij}}=\\
        &= \sum_{i=0}^n\sum_{j>i}\frac{2}{j-i+1} \leq\\
        &\leq n \underbrace{\sum_{k=1}^{n} \frac{2}{k}}_{
            \substack{\text{numero}\\ \text{armonico}}
        } = 2n \log n + O(n)
    \end{align*}
    Per cui il numero atteso di confronti nell'esecuzione del \emph{QuickSort} probabilistico è $O(n\log n)$.
    
    \vspace{1em}\noindent
    Si passa ora all'analisi dell'algoritmo deterministico, ponendo come premessa un input $S$ perfettamente scelto da tutte le permutazioni di elementi. Ponendo questa premessa, se tutte le permutazioni hanno la stessa probabilità (quindi uniforme) allora anche il sottoinsieme $[s_i,\dots,s_j]$ ha probabilità uniforme.

    L'analisi dell'algoritmo in questo particolare caso è analoga alla versione probabilistica, si ottiene tempo di esecuzione $O(n \log n)$. È importante notare che nel caso deterministico è necessario che l'input sia distribuito in modo uniforme, ciò è un'assunzione molto forte.
    
    \subsection{Algoritmo per il Calcolo della Mediana}
    %TODO: Inserire sezione richiami di probabilità con esempi (cupons collector)
    Come visto in \textcolor{red}{Sezione 0} la mediana è un indice statistico. Dato $X$ un insieme non ordinato di elementi restituirene la mediana. Un algorimo deterministico banale consiste nell'ordinare $X$ e restituire l'elemento corretto, per una complessittà di $O(n\log n)$. Esiste anche un algoritmo deterministico che opera in $O(n)$.

    \subsubsection{Algoritmo probabilistico}
    L'algoritmo probabilistico prende in input $S$ un insieme di $2k +1$ elementi da un'universo totalmente ordinato e restituisce in outupt \emph{$k+1$-esimo} elemento nell'insieme.
    
    L'algoritmo opera come segue
    \begin{enumerate}
        \item Scegli un \emph{multi-insieme} $R$ di $n^{3/4}$ elementi presi da $S$, Ordina $R$
        \item Sia $d$ l'elemento $\left(\frac{1}{2}n^{3/4} - \sqrt{n}\right)$\emph{-esimo} più piccolo in $R$ ordinato
        \item Sia $u$ l'elemento $\left(\frac{1}{2}n^{3/4} + \sqrt{n}\right)$\emph{-esimo} più piccolo in $R$ ordinato
        \item Calcola l'insieme $C$ di elementi compresi tra $d$ e $u$, inoltre si definiscono $\ell_d,\ell_u$
        \[
            C =\{ x \in S \mid d \leq x \leq u \} \quad \ell_d = |\{ x \in S \mid x < d\}| \quad \ell_u = |\{ x \in S \mid x > u\}|
        \]
        \item Se $\ell_d > \frac{n}{2}$ o $\ell_u > \frac{n}{2}$ termina l'esecuzione con errore
        \item Se $|C| < 4n^{3/4}$ ordina $C$
        \item Restituisci in output l'elemento $\left( \lfloor \frac{n}{2}\rfloor - \ell_d +1\right)$\emph{-esimo} in $C$ ordinato.
    \end{enumerate}
    Facciamo chiarezza sull'algoritmo spiegando più nel dettaglio i diversi punti. Partiamo con la costruzione di $R$ da $S$.
    \begin{figure}[h]
  \centering
  \begin{minipage}{.42\linewidth}
    \begin{algorithm}[H]
    \setlength{\algomargin}{2em}
    \caption{Costruzione di $R$}
    \label{alg:median}
    $R \gets \emptyset$\;
    \For{$i = 0$ \KwTo $n^{3/4}$}{
        Prendi $u \in_u S$ \textit{($u$ è mantenuto in $S$)} \;
        $R \gets R \cup \{u\}$\;
    }
    \end{algorithm}
\end{minipage}
\end{figure}


    La probabilità che un elemento $u$ non finisca in $R$ in una estrazione è $1 - \frac{1}{n}$, per $k$ estrazioni indipendenti si ha $\left( 1- \frac{1}{n}\right)^{k}$ ovvero la probabilità di non inserire $u$ in $k$ estrazioni successive. Per cui la probabilità che un qualsiasi elemento $u$ di $S$ sia finito in $R$ è $1 - \left( 1- \frac{1}{n}\right)^{n^{3/4}}$.
    \[
        \forall u \in S \ \Pr{[u \in R]} = 1 - \left( 1- \frac{1}{n}\right)^{n^{3/4}}\approx 1 - e^{-\frac{1}{n^{3/4}}} \approx \frac{1}{n^{1/4}}
    \]
    Quindi si ottiene un \emph{multi-insieme} $R$ che rappresenta un ``campione'' uniforme di $S$, e due elementi $d,u$ in cui sono inclusi circa $2\sqrt{n}$ valori intermedi, per cui gli elementi in $R$ sono distribuiti circa uniformemente tra gli elementi di $S$.
    Si ha l'insieme $C$ per cui $|C| < \frac{n}{\log n}$, per cui è possibile ordinare $C$ in tempo lineare.
    \begin{center}
    \begin{tikzpicture}[scale=1.2, every node/.style={font=\small}]

    \foreach \i in {1,2,...,39}
        \draw[gray!40, very thin] ({-5 + \i*(10/40)}, 0.08) -- ({-5 + \i*(10/40)}, -0.08);

    \foreach \x in {-3, 0, 3}
        \draw[thick] (\x, 0.18) -- (\x, -0.18);

    \node[above=6pt] at (-3,0) {$d$};
    \node[above=6pt] at (3,0) {$u$};
    \node[above=6pt] at (0,0) {mediana di $S$};

    \draw[gray, thin, dashed] (-5,-0.1) -- (-5,-0.45);
    \draw[gray, thin, dashed] (-3,-0.2) -- (-3,-0.45);
    \draw[gray, thin, dashed] (3,-0.2) -- (3,-0.45);
    \draw[gray, thin, dashed] (5,-0.1) -- (5,-0.45);

    % Graffa sinistra (da inizio a d)
    \draw[decorate, decoration={brace, mirror, raise=15pt, amplitude=6pt}]
        (-5,0) -- (-3,0)
        node[midway, below=22pt] {$\ell_d$};

    % Graffa destra (da u a fine)
    \draw[decorate, decoration={brace, mirror, raise=15pt, amplitude=6pt}]
        (3,0) -- (5,0)
        node[midway, below=22pt] {$\ell_u$};


    % Disegno dell'Array S
    \draw[thick] (-5,0) -- (5,0);
    \draw[thick] (-5,0.25) -- (-5,-0.25);
    \draw[thick] (5,0.25) -- (5,-0.25);
    \node[left=3pt] at (-5,0) {$S$};
    \end{tikzpicture}
\end{center}


    Con alta probabilità $\frac{1}{2}n^{3/4} - \sqrt{n}$ elementi sono minori della mediana, mentre $\frac{1}{2}n^{3/4} - \sqrt{n}$ elementi sono maggiori della mediana, dato questo, allora la mediana appartiene al sottoinsieme $C$, se queste condizioni sono vere la mediana si trova proprio in posizione $\lfloor \frac{n}{2} \rfloor - \ell_d + 1$ di $C$ ordinato. 

    \vspace{1em}\noindent
    L'algoritmo come visto sopra ha anche dei casi di fallimento, si dà una forma più rigorosa agli eventi che portano l'algoritmo al fallimento. Si definiscono le variabili aleatorie $Y_1,Y_2$. $Y_1$ conta il numero di elementi in $R$ (campioni) soto la mediana, al contrario $Y_2$ conta il numero di elementi sopra la mediana. Formalmente si definisce $y_{1}^i$ la variabile binaria 
    \[
        y_{1}^i = \begin{cases*}
            1 \text{ se } x_i \in R < Mediana\\
            0 \text{ altrimenti}
        \end{cases*}
        \quad
        Y_1 = \sum_{i = 1}^{n^{3/4}}y_{1}^i
    \]  
    Si definiscono tre eventi $E_1$ per cui $C$ (l'intervallo di elementi $[d,u]$) è spostato a eccessivamente sinistra, l'evento $E_2$ è simmetrico, l'evento $E_3$ rappresenta una scelta di elementi in $R$ non sufficientemente uniformeme. 
    \begin{align*}
        &E_1: Y_1 < \frac{1}{2}n^{3/4}-\sqrt{n}\\
        &E_2: Y_2 < \frac{1}{2}n^{3/4}-\sqrt{n}\\
        &E_3: |C| > \frac{n}{\log n}
    \end{align*}
    È necessario uno solo di questi eventi per il fallimento dell'algoritmo, si calcolano le probabilità di avvenimento. Lo spazio di campionamento e l'insieme di tutte le possibili scelte di $n^{3/4}$ elementi da $S$ con ripetizione, ovvero $n^{n^{3/4}}$ punti. Per calcolare la probabilità dell'evento $E_1$ (legato alla variabile aleatoria $Y_1$),
    \[
        \E{y_1^i} = \frac{1}{2}
        \quad
        \Var{[y_1^i]} = \frac{1}{4}
    \]
    Inoltre 
    \begin{align*}
        &\Pr{[y_1^i = 1] = \frac{1}{2}}\\
        &\E{Y_1} = \E{\sum_{i=0}^{n^{3/4}}{y_1^i}} = \sum_{i=0}^{n^{3/4}}{\E{y_1^i}} = \frac{1}{2}n^{3/4}
    \end{align*}
    Si ottiene un risultato simile per la varianza $\Var{[y_1^i]} = \frac{1}{4}$. 

    \vspace{1em}\noindent
    Per la disuguaglianza di Chebyshev si ha
    \begin{align*}
        \Pr{[E_1]} &= \Pr{\left[Y_1 < \frac{1}{2}n^{3/4} - \sqrt{n} \right]} \leq\\
        &\leq \Pr{\left[|Y_1 - \E{Y_1}| > \sqrt{n}\right]} \leq\\
        &\leq \frac{\Var{[Y_1]}}{\left(\sqrt{n}\right)^2} = \boxed{\frac{1}{4n^{1/4}}}
    \end{align*}
    Per l'evento $E_2$ si ottiene un risultato simile $\Pr{[E_2]} \leq \frac{1}{4n^{1/4}}$, infine con alta probabilità $\Pr{[E_1 \cup E_2]} \leq \boxed{\frac{2}{4n^{1/4}}}$.

    \vspace{1em}\noindent
    Così facendo si è posto un \emph{upper-bound} alla probabilità del verificarsi dei primi due eventi $E_1,E_2$, per calcolare la probabilità di $E_3$ si definiscono altri due eventi $\mathcal{E}_{3,1}$ $\mathcal{E}_{3,2}$ che si verificano rispettivamente quando almeno $2n^{3/4}$ elementi in $C$ sono più grandi e più piccoli della mediana.
    \begin{align}
        \mathcal{E}_{3,1} = |\{ x \in C \mid x > M \}| > 2n^{3/4} \quad
        \mathcal{E}_{3,2} = |\{ x \in C \mid x < M \}| > 2n^{3/4}
        \label{event:1}
    \end{align}
    Si dimostra facilmente che per $|C| > 4n^{3/4}$ almeno uno dei due eventi è verificato, poiché si assume che la mediana $M$ sia in $C$. Prendiamo in considerazione l'evento $\mathcal{E}_{3,2}$, si vuole calcolare il numero di elementi in $R$ fuori da $C$, ricordando che $|R| = n^{3/4}$
    \[
        |\{ x \in R \mid x > u\}| = n^{3/4} - \left(\frac{1}{2}n^{3/4}+\sqrt{n}\right) = \frac{n^{3/4}}{2} - \sqrt{n}
    \]
    Inoltre dalla definizione di $\mathcal{E}_{3,2}$ (\ref{event:1}) allora $u$ è almeno l'elemento \emph{$\frac{1}{2}n+2n^{3/4}$-esimo} più grande in $S$ ($2n^{3/4}$ sono gli elementi più grandi della mediana $M$). Per cui almeno $\frac{1}{2}n^{3/4} - \sqrt{n}$ campioni di $R$ sono stati selezionati tra i $\frac{1}{2}n - 2n^{3/4}$ più grandi in $S$. 

    \vspace{1em}
    Sia ora $X$ la variabile aleatoria che conta il numero di campioni tra gli $\frac{1}{2}n - 2n^{3/4}$ elementi più grandi in $S$
    \begin{gather}
        X_i = \begin{cases*}
            1 \text{ l'elemento \emph{i-esimo} campiona tra i più grandi}\\
            0 \text{ altrimenti}
        \end{cases*}\\
        X = \sum_{i=0}^{n^{3/4}}X_i
    \end{gather}
    Quindi banalmente per $1/2n - 2n^{3/4}$ numero di elementi favorevoli, su $n$ numero di elementi totali
    \[
        \Pr{[X_i = 1]} = \frac{1/2n - 2n^{3/4}}{n}
    \]
    Inoltre si calcolano i valori attesi e varianze
    \begin{gather*}
        \E{X_i} = \E{(X_i)^2} = \frac{1}{2}- \frac{2}{n^{1/4}}\\
        \Var{[X_i]} = \E{(X_i)^2} - \E{X_i}^2 \leq \frac{1}{4}\\
        \E{X} = n^{3/4} \left( \frac{1}{2}- \frac{2}{n^{1/4}}\right)=\frac{1}{2}n^{3/4} - 2\sqrt{n}\\
        \Var{[X]}\leq \frac{1}{4}n^{3/4}
    \end{gather*}
    Facendo uso della disuguaglianza di Chebyshev si da un \emph{bound} alla probabilità dell'evento $\mathcal{E}_{3,1}$
    \begin{align*}
        \Pr{[\mathcal{E}_{3,1}]} &= \Pr{[X \geq \frac{1}{2}n^{3/4} - \sqrt{n}]}\\
        &\leq \Pr(|X- \E{X}| \geq \sqrt{n})\\
        &\leq \frac{\Var[X]}{n} = \frac{\frac{n^{3/4}}{4}}{n} = \frac{1}{4n^{1/4}}.
    \end{align*}
    In modo del tutto simile $\Pr{\left[\mathcal{E}_{3,2}\right]}\leq \frac{1}{4n^{1/4}}$ tornando alla probabilità dell'evento principale $E_3$
    \[
        \Pr{[E_3]} \leq \Pr{[\mathcal{E}_{3,2}]} + \Pr{[\mathcal{E}_{3,2}]} = \boxed{\frac{1}{2n^{1/4}}}.
    \]
    Concludendo la probabilità fallimento e successo dell'algoritmo sono
    \begin{align*}
        \Pr{[\text{Fallimento}]} &= \Pr{[E_1]} + \Pr{[E_2]} + \Pr{[E_3]} \\
        &\leq \frac{2}{4n^{1/4}} + \frac{1}{2n^{1/4}} = \frac{1}{n^{1/4}}\\
        \Pr{[\text{Successo}]} &= 1 - \Pr{[\text{Fallimento}]} \geq 1 - \frac{1}{n^{1/4}}.
    \end{align*}
    \section{Altri algoritmi probabilistici}
    \subsection{Contention Resolution in sistemi distribuiti}
Dai $n$ processi $P_1,\dots,P_n$ ognuno dei quali compete per l'accesso a un database condiviso. Se due o più processi accedono al database contemporaneamente, tutti i processi sono bloccati. Determinare un protocollo per assicurare a ogni processo l'accesso al database. \emph{Nota: I processi non possono comunicare tra di loro}.

\vspace{1em}\noindent
\emph{Protocollo:} Ogni processo richiede l'accesso al database al tempo $t$ con probabilità $p = \frac{1}{n}$.

Si definisce $S_{i,t}$ l'evento in cui il processo \emph{i-esimo} riesce ad accedere al database al tempo $t$. 
\begin{theorem}
    Siano $S_{i,t}$ le variabili aleatorie definite come sopra
    $\frac{1}{en} \leq \Pr[S_{i,t}] \leq \frac{1}{2n}$.
\end{theorem}

\begin{proof}
    Ogni processo è indipendente, per cui per l'indipendenza $\Pr[S_{i,t}] = p(1-p)^{n-1}$, in cui $p$ è la probabilità di successo di un solo processo, e $(1-p)^{n-1}$ è la probabilità in cui nessuno degli altri $n-1$ processi richiede l'accesso.

    \vspace{1em}
    Per cui impostando $p=\frac{1}{n}$ si ha che 
    \[
        \Pr\left[S_{i,t}\right] = \frac{1}{n}\left(1-\frac{1}{n}\right)^{n-1} \approx \boxed{\frac{1}{en}}
    \]
    \emph{Nota:} $\left( 1 - \frac{1}{n}\right)^{n-1}$ converge a $\frac{1}{e}$ per $n \to \infty$.
\end{proof}
% TODO: Riformulare
\begin{theorem}
    La probabilità che l'evento \emph{i-esimo} fallisca ad accedere al database in $en$ tentativi $\leq \frac{1}{e}$. Dopo $en(c \log n)$ tentativi, la probabilità $\leq \frac{1}{n^c}$
\end{theorem}
\begin{proof}
    Sia $F_{i,t}$ l'evento in cui il processo $i$ fallisce fallisce nell'accesso al database nei round $1,\dots,t$. Poiché i tentativi sono indipendenti tra di loro, si ha che 
    \[
        \Pr[F_{i,t}] \leq \left(1- \frac{1}{en}\right)^t
        \]
        Per cui scegliendo $t = \lceil en \rceil$ si ha 
        \[
            \Pr[F_{i,t}] \leq \left(1-\frac{1}{en}\right)^{\lceil en \rceil} \leq \left(1-\frac{1}{en}\right)^{en} \leq \frac{1}{e}
            \]
            Scegliendo $t = \lceil en \rceil\cdot\lceil c \log n \rceil$ si ha 
            \[
                \Pr[F_{i,t}] \leq \left(\frac{1}{e}\right)^{c \log n} = \frac{1}{n^c}
            \]
\end{proof}
Dati questi teoremi come basi, è possibile definire la probabilità per cui tutti i processi abbiano successo in $2e\, n \log n$ tentativi
$\Pr[\text{Successo}] \geq 1 - \frac{1}{n}$

\begin{proof}
    Sia $F_t$ l'evento in cui almeno uno degli $n$ processi fallisce nell'accesso nei tentativi $1,\dots,t$
    \begin{align*}
        \Pr[F_t] &= \Pr\left[ \bigcup_{i=1}^{n} F_{i,t} \right] \leq\\
        (\substack{\text{Per \emph{union buond \ref{union}}}}) &\leq \sum_{i=1}^n \Pr[F_{i,t}] \leq n\left(1-\frac{1}{en}\right)^t
    \end{align*}
    Inoltre per $t = 2e \, n \log n$ tentativi ($c = 2$), per il teorema precedente
    \begin{align*}
        \Pr[F_{i,t}] \leq n \cdot \left(\frac{1}{e}\right)^{2 \log n} = \frac{1}{n^2}.
    \end{align*}
    Concludendo \[
        \Pr[F_t] \leq n \cdot \frac{1}{n^2} = \frac{1}{n}
    \]
\end{proof}
% TODO: Rifinire
\subsection{Problema del Load balancing}
Si ha un sistema in cui $m$ \emph{jobs} arrivano in stream e devono essere processati immediatamente su $n$ processori identici. È necessario trovare un assegnamento che bilanci il \emph{carico di lavoro} tra i processori.

Un approccio semplice e deterministico è quello di avere un \emph{controller} centralizzato che distribuisca il carico con uno \emph{scheduling} Round-Robin. Ogni processore riceve al massimo $\lceil \frac{m}{n} \rceil$ \emph{jobs}. 

Per un approccio decentralizzato si fa uso della probabilità e del calcolo probabilistico, ogni \emph{job} viene assegnato a un processore in modo uniformemente casuale. Si calcola la probabilità che un processore riceva troppi \emph{jobs} $\left(> \lceil \frac{m}{n} \rceil \right)$.

A tale scopo si definiscono le variabili aleatorie $X_i$ il numero di \emph{jobs} assegnati al processore $i$,
\[
    Y_{ij} = \begin{cases*}
        1 \text{ se il job $j$ è assegnato a $i$}\\
        0 \text{ altimenti}
    \end{cases*}
\]

Per cui si hanno i dati $\E{Y_{ij}} = \frac{1}{n}, \Pr[Y_{ij} = 1] = \frac{1}{n}$. Inoltre $X_i = \sum{j=1}^{n}Y_{ij}$ e si definisca $\mu = \E{X_i} = 1$, rappresenta il carico \emph{atteso} del processore $i$.

Applicando i Chernoff Bounds da \ref{chernoff}, con $\delta = c - 1$ si ha
\[
    \Pr[X_i > c] < \frac{e^{c-1}}{c^c}
\]

Sia $\gamma(n)$ il numero $x$ tale che $x^x = n \implies (\gamma(n)^{\gamma(n)} = n)$ e sia $c = e\cdot \gamma(n)$
\begin{align*}
    &\Pr[X_i > c] < \frac{e^{c-1}}{c^c} < \left( \frac{e}{c}\right)^{c} =\\
    &=\left( \frac{1}{\gamma(n)}\right)^{e\gamma(n)} < \left( \frac{1}{\gamma(n)} \right)^{2\gamma(n)} = \frac{1}{n^2}.
\end{align*}

Per lo union bound \ref{union} si mostra il risultato
\[
    \Pr\left[\bigcup_{j = 1}^{n} Y_{ij}\right] < \Pr\left[{\sum_{j=1}^{n}Y_{ij}}\right] = \Pr[X_i] > \frac{1}{n^2}
\]
Per cui $\lnot\Pr{\left[\sum_{i = 1}^nX_i\right]} = 1 - n\Pr\left[ \bar{X}_i  \right] < 1 - \frac{1}{n}$ è la probabilità che \emph{nessun processore riceve più di $e\cdot\gamma(n)$ jobs}. $e\cdot\gamma(n) = \Theta\left(\frac{\log n}{\log \log n} \right)$.

\begin{theorem}
    Si supponga il numero di \emph{jobs} $m = 16 n \log n$. Allora in media ognuno degli $n$ processori gestisce $\mu = 16 \log n$ \emph{jobs}, con alta probabilità ogni processore gestirà tra $x$ compreso tra $\frac{1}{2}$ e $2$ carico medio.
\end{theorem}

\begin{proof}
    Siano $X_{i}, Y_{ij}$ variabili aleatorie definite come sopra, applicando il Chernoff bound con $\delta = 1$
    \begin{align*}
        &\Pr[X_i > 2\mu] < \left( \frac{e}{4}\right)^{16 n \log n} < \left( \frac{1}{e}\right)^{\log n} = \frac{1}{n^2}\\
        &\Pr\left[X_i > \frac{1}{2}\mu\right] < e^{-\frac{1}{2}\left(\frac{1}{2}\right)^2{16 n \log n}} = \frac{1}{n^2} 
    \end{align*}
    Per lo union bound \ref{union} ogni processore ha il carico tra $\frac{1}{2}$ e $2$ del carico medio, con probabilità $\geq 1- \frac{2}{n}$
\end{proof}
    \section{Hashing}
    In questa sezione vedremo le funzioni hash, delle funzioni che permettono di mappare degli elementi di un universo in uno degli elementi di un insieme finito. 
Una funzione hash è una funzione $h: U \to [0,n)$, informalmente $h$ è utilizzata per ``randomizzare'' i dati:
\begin{enumerate}
    \item $h(x)$ dovrebbe essere più casuale possibile. Idealmente dovrebbe mappare in modo più uniforme possibile
    \item $h(x)$ dovrebbe essere veloce da calcolare. Idealmente proporzionale al tempo di accesso a $x$
    \item $h$ dovrebbe occupare poco spazio in memoria. Idealmente $O(1)$ parole di memoria
\end{enumerate}
\begin{definition}[Famiglia di funzioni hash]
    Una famiglia di funzioni hash $\mathcal{H}$ è un sottoinsieme di funzioni $h: U \to [0,n) $, nell'inieme di tutte le funzioni hash $\mathcal{H} \subseteq [0,n)^{U}$.
\end{definition}
\vspace{0.5em}
\begin{definition}[Funzione hash uniforme]
    Una funzione hash $h$ da un universo $U = \{x_0,\dots,x_u\}\ (u = |U|)$ . $\mathcal{H}$ è detta uniforme se per ogni $y_1,\dots,y_u \in [0,n)$ si ha che
    \[
        \Pr\left[h(x_1),\dots,h(h_u) = y_1,\dots,y_u\right] = \frac{1}{n^u}
    \]
\end{definition}
\begin{definition}
    Una famiglia $\mathcal{H}$ è detta \emph{k-indipendente} se e solo se per una scelta uniforme di $h \in \mathcal{H}$ si ha 
    \begin{enumerate}
        \item Per ogni $x \in U$, $h(x)$ è una variabile aleatoria uniforme in $[0,n)$
        \item Le variabili aleatorie $h(1),h(2),\dots,h(u)$ sono \emph{k-indipendenti}
    \end{enumerate}
    o in modo equivalente
    \[
        \Pr\left[\bigcap_{i=1}^{k}h(x_i) = y_i\right] = \frac{1}{n^k}
    \]
    per ogni scelta di $x_1,\dots,x_k \in [1,u]$ $y_1,\dots,y_k \in [0,n)$. O anche la \emph{k-tupla} $(h(x_1),\dots, h(x_k))$ è uniforme in $[0,n)^k$. Per $k=n$ la famiglia di funzioni hash è detta completamente uniforme.
\end{definition}

\subsection{Hashing universale}
\begin{center}
    \emph{Nota Bene: In questi appunti, $n$ è definito come il numero di slot di arrivo della funzione hash.}
\end{center}
Una famiglia di funzioni hash $\mathcal{H}$ è detta universale se scegliendo $h \in_u \mathcal{H} \ \ h: U \to [0,n)$ dati $\forall x_1 \neq x_2 \in U$
\[
    \Pr[h(x_1) = h(x_2)] \leq \frac{1}{n}
\]in pratica la probabilità di collisione di due elementi di $U$.
Si nota che questa probabilità è quella che ci si aspetta per una funzione hash che dia un risultato \emph{veramente} totalmente casuale

\begin{theorem}[2-indipendenza implica universalità]
    \begin{align*}
        \Pr[h(x_1) = h(x_2)] &= \sum_{y \in U} \Pr[h(x_1) = h(x_2) \land h(x_2) = y]\\
        &= \sum_{y \in U} \Pr[h(x_1) = y \land h(x_2) = y]\\
        &= \sum_{y \in U} \frac{1}{n^2} = \frac{1}{n}
    \end{align*}
\end{theorem}
\begin{theorem}
    Sia  $\mathcal{H}$ una famiglia di funzioni hash universali, $S \subseteq U$ un insieme di $k$ elementi. Sia $u \in S$. Si sceglie in modo uniforme una funzione $h$ da $\mathcal{H}$, e sia $X$ la variabile aleatoria che conta il numero di elementi di $S$ mappati nello stesso elemento $h(u)$ allora
    \[
        \E{X} \leq 1 + \frac{k}{n}.
    \]
\end{theorem}
\begin{proof}
    Sia $u$ fissato, per ogni $s \in S$, si definisce la variabile aleatoria $X_s$ associata
    \[
        X_s = \begin{cases*}
            1 \text{ se } h(s) = h(u)\\
            0 \text{ altrimenti}
        \end{cases*} 
        \quad X = \sum_{s \in S}X_s
    \]
    Per cui si ha
    \begin{align*}
        &\E{X} = \sum_{s \in S}{\E{X_s}} = \sum_{s \in S}\Pr[h(s) = h(u)]\\
        &= 1 + \sum_{s \in S-\{u\}}\Pr[h(s) = h(u)]\\
        \substack{\text{(Per universalità)}} &\leq 1 + \frac{k}{n}.
    \end{align*}
    \emph{Nota: per $m = \Theta(m)$ si ha $O(1)$ tempo per operazione.}
\end{proof}

Si passa ora alla definizione di una famiglia di funzioni hash universale
\subsection{Esempi famiglie hash universali}
\subsubsection{Esempio 1}
Si sceglie un numero primo $m > n$, è noto che esiste $m$ tale che $\forall n \ n \leq m \leq 2n$, successivamente si identifica ogni elemento $x \in U$ come un intero in \emph{base m}, di $r$ cifre $x = (x_1,x_2,\dots,x_r)$. Per una $a = (a_1,a_2,\dots,a_n) \in U, a_i \in [m]$ fissata si definisce
\begin{align}
    h_a(x) = \left[ \sum_{i=1}^{r}a_ix_i\right] \mod m
\end{align}
Per cui si definisce la famiglia di funzioni hash universale $\bar{\mathcal{H}}$
\[
 \bar{\mathcal{H}} = \{ h_a : a \in U\}
\]
Per $|U| = n$, $r$ deve essere $\log_m(n)$ poiché $m^r \geq n$. Ne risulta che per scegliere una funzione hash in modo uniforme $h \in_u \bar{\mathcal{H}}$ è sufficiente scegliere $a \in_u U$.

\begin{theorem}[Universalità]
    La famiglia $\bar{\mathcal{H}}$ è una famiglia universale
\end{theorem}

\begin{proof}
    Sia $x = (x_1,x_2\dots,x_r)$ e $y = (y_1,y_2\dots,y_r)$ $\in U$ tale che $x \neq y$. 
    
    Si deve dimostrare $\Pr[h(x) = h(y)] \leq \frac{1}{n}$ (universalità).
    \vspace{1em}
    Poiché $x\neq y$ esiste $y$ t.c. $x_j \neq y_j$.
    $h_a(x) = h_a(y)$ se e solo se
    \[
        a_j\underbrace{(y_j-x_j)}_z = \underbrace{\sum_{i\neq j}a_i(x_i - y_i)}_\alpha \mod m
    \]
    Si assume $a \in_u U$, quindi si assume $a_i$ fissato $\forall i \neq j$. Per $m$ primo $\mathbb{Z}_m$ è un campo, per cui, per $z \neq 0$ esiste un'unica inversa moltiplicativa $z^{-1}$ tale che $z \cdot z^{-1} = 1 \mod m$

    \begin{align*}
        a_j \cdot z \cdot z^{-1} &= \alpha \cdot z^{-1} \mod m\\
        a_j &= \alpha \cdot z^{-1} \mod m \substack{\text{ (soluzione unica)}} \\
        &\Pr[a_j \equiv_m \alpha z^{-1}] \leq \frac{1}{m}.
    \end{align*}
\end{proof}

\subsubsection{Esempio 2}
Si mostra ora una famiglia  di funzioni hash \emph{2-indipendenti} sia $p \geq n$ un numero primo, siano $a,b \in \mathbb{Z}_p$ definite come nell'esempio precedente, si definisce la funzione hash $h_{a,b}$
\[
    h_{a,b}(x) = \left[ax + b \mod p\right] \mod m
\]
in pratica la funzione hash è definita come un polinomio scelto uniformemente in $\mathbb{Z}_p$ di grado 1. Si definisce la famiglia $\hat{\mathcal{H}}$
\[
    \hat{\mathcal{H}} = \{ h_{a,b} \mid a,b \in \mathbb{Z}_p\}
\] 
\begin{theorem}[2-indipendenza e universalità]
    $\hat{\mathcal{H}}$ è una famiglia \emph{2-indipendente} e universale
\end{theorem}
\begin{proof}
    Sia $X = (ax + b) \mod p$ e $Y = (ay +b) \mod p$ per $x \neq y$. Poiché $a \neq 0$ e $p > n$ allora $X \neq Y$.
    \[
        h_{a,b}(x) = h_{a,b}(y) \iff X = Y \mod m
    \]
    \begin{enumerate}
        \item $X$ e $Y$ sono distribuite in modo uniforme su $\mathbb{Z}_p$, poiché $a,b$ sono, a loro volta, distribuite in modo uniforme e $h$ è lineare iniettiva.
        \begin{proof}
            \begin{align}
                    ax + b \equiv_p ay + b
                &\iff
                    a(x-y) \equiv_p 0
                \\
                \implies (x-y) \equiv_p 0
                &\iff x = y
            \end{align}
        Il passaggio (6) è giustificato poiché $ x,y < p-1$
        \end{proof}
        
        \item $X$ e $Y$ sono quasi indipendenti tra loro, $\Pr[X = i \land Y = j] = \frac{1}{(p-1)p}$
        \begin{proof}
            \begin{equation}
                \begin{cases*}
                    ax + b \equiv_p i\\
                    ay + b \equiv_p j\\
                \end{cases*}
            \end{equation}
            Esiste un unica soluzione per $(a,b) \in {\mathbb{Z}_p}^2$, poiché il rango è 2.
            Siano $f,g$ soluzioni uniche del sistema lineare 
            \begin{align*}    
                \Pr\left[ (ax + b) \equiv_p i \land (ay + b) \equiv_p j\right] &= Pr[a = f(x,y,i,j) \land b = g(x,y,i,j) ]\\
                &= \Pr[a = f(x,y,i,j)] \cdot \Pr[b = g(x,y,i,j)]
            \end{align*}
            Per cui
            \begin{align*}    
                \Pr\left[ Y = j\mid X = i\right] &= \frac{\Pr[X = i \land Y = j]}{\Pr[X = i]}\\
                &= \frac{1}{p-1}
            \end{align*}    
        \end{proof}    
    \end{enumerate}
    Per una $i$ fissata in $\mathbb{Z}_p$ ci sono al massimo $\frac{p}{m}-1 \leq \frac{(p-1)}{m}$ valori per $Y$ tale che $Y = i \mod m$, ovvero tutti gli interi $\in \mathbb{Z}_p$ per cui la distanza da $i$ è un multiplo di $m$. Da 2 $\Pr[Y = j \mid X = i] = \frac{1}{p-1}$ per l'union bound si ottiene $\Pr[Y=i \mod m \mid X = i] \leq \frac{p-1}{m} \frac{1}{p-1} = \frac{1}{m}$. $\hat{\mathcal{H}}$ è universale.
\end{proof}

\subsection{Perfect hashing}
Parliamo ora di funzioni hash \emph{perfette} ovvero una funzione hash senza collisioni
\begin{definition}[funzione hash perfetta]
    Una funzione hash $h: [1,n] \to [0,M]$ è detta perfetta su un insieme $A \subseteq [1,n]$ se e solo se per ogni $x_1 \neq x_2 \in A$ si ha $h(x_1) \neq h(x_2)$, quindi è iniettiva su $A$. In generale cerchiamo una funzione che abbia tale proprietà con alta probabilità.
\end{definition}
\begin{theorem}
    Se una famiglia $\mathcal{H}$ di funzioni $h: [1,n] \to [0,M)$ è universale e $M \geq n^{c+2}$ per una costante $c$ arbitrariamente grande allora $h \in_u \mathcal{H}$ è perfetta su ogni insieme $A \leq [1,n]$ con alta probabilità.
\end{theorem}
\begin{proof}
    Universalità significa che $\Pr[h(x_1) = h(x_2)] \leq \frac{1}{M}$ per $x_1 \neq x_2$, ci sono al massimo $|A|^2 \leq n^2$ coppie di elementi distinti in $A$, dallo union bound la probabilità di avere almeno una collisione è al massimo $\frac{n^2}{M}$, scegliendo $M = n^{c+2}$ si ha una collisione con probabilità $\leq n^{-c}$, ovvero la funzione è perfetta con alta probabilità $(1-n^{-c}$).
\end{proof}

Il problema del dizionario è spesso presentato come l'esempio pratico per mostrare l'efficacia delle funzioni hash
\subsection{Il problema del dizionario}
Il dizionario è un tipo di dati, dato un universo $U$ di elementi possibili, mantiene un sottoinsieme arbitrario $S \subseteq U$ tale che operazioni come intersezione, unione, e ricerca in $S$ siano efficienti, queste operazioni sono rappresentate come:
\begin{itemize}
    \item \texttt{create()} inizializza un dizionario vuoto
    \item \texttt{insert(u)} aggiunge un elemento $u \in U$ a $S$
    \item \texttt{delete(u)} rimuove un elemento $u$ da $S$
    \item \texttt{lookup(u)} risponde alla domanda $s \in S?$
\end{itemize}

La sfida principale riguarda la grandezza dell'universo $U$ che può essere estremamente grande, per cui definire un array di dimensione $|U|$ non è ragionevole, si cerca una soluzione proporzionale al sottoinsieme $n := |S|$.
Una soluzione deterministica è implementata con gli alberi \emph{AVL} con $O(n)$ spazio e $O(\log n)$ tempo per operazione.

\vspace{1em}
Si presenta una soluzione probabilistica con $O(n)$ spazio e $O(1)$ tempo atteso per operazione, a tale scopo si presenta il concetto di tabelle hash.
\subsubsection{Tabelle hash}
Viene creato un array $H$ di grandezza $m \approx n$, quindi $H = [m]$. Si ha una collisione quando $h(u) = h(v)$ per u $\neq v$, mediamente per il paradosso del compleanno una collisione è attesa ogni $\sqrt{n}$ inserimenti.

Per ogni posizione $i$ dell'array $H[i]$ contiene delle \emph{linked list} che contengono tutti gli elementi che collidono, solitamente chiamate \emph{liste di trabocco}.

% Diagramma TikZ: Funzionamento di una Hash Table con Liste di Trabocco (Chaining)
% Richiede: \usepackage{tikz} \usetikzlibrary{shapes.geometric, arrows.meta, positioning}
\begin{figure}[h]
    \centering
    \begin{tikzpicture}[
        scale=0.9,
        font=\small,
    % Definizione Colori
    colorU/.style={draw=black!80, fill=gray!5},
    colorS/.style={draw=blue!60!black, fill=blue!10},
    colorH/.style={draw=black!80!black, fill=black!5},
    colorArrow/.style={draw=gray!40!black},
    colorList/.style={draw=blue!60!black, fill=white},
    % Stili
    universo/.style={universo_shape, colorU, minimum width=4.5cm, minimum height=6.5cm, label={above:\textcolor{gray!80!black}{$U$}}},
    sottoinsieme/.style={universo_shape, colorS, minimum width=2.8cm, minimum height=4cm, label={above:\textcolor{blue!80!black}{$S \subseteq U$}}},
    cella/.style={draw=black!80!black, minimum width=1.4cm, minimum height=0.75cm, fill=white, outer sep=0pt},
    % Indici a sinistra
    indice/.style={font=\scriptsize\ttfamily\color{gray!80!black}, anchor=east, xshift=-0.1cm},
    funzione/.style={->, >=Stealth, thick, colorArrow},
    lista/.style={draw=blue!60, minimum width=0.8cm, minimum height=0.6cm, fill=blue!5, inner sep=2pt},
    universo_shape/.style={draw, ellipse},
    puntamento/.style={->, >=Stealth, semithick, blue!80!black}
]

    % --- Rappresentazione degli Insiemi ---
    \node[universo] (U) at (0,0) {};
    \node[sottoinsieme] (S) at (0,-0.5) {};
    
    % Chiavi all'interno di S
    \node[circle, fill=blue!80!black, inner sep=1.8pt, label={[text=blue!90!black]left:{$k_1$}}] (k1) at (-0.5, 0.5) {};
    \node[circle, fill=blue!80!black, inner sep=1.8pt, label={[text=blue!90!black]left:{$k_2$}}] (k2) at (0.4, -0.8) {};
    \node[circle, fill=blue!80!black, inner sep=1.8pt, label={[text=blue!90!black]left:{$k_3$}}] (k3) at (-0.2, -2) {};
    \node[circle, fill=blue!80!black, inner sep=1.8pt, label={[text=blue!90!black]left:{$k_4$}}] (k4) at (0.2, 0.8) {};

    % --- Rappresentazione dell'Array H ---
    \begin{scope}[shift={(6.5,2)}]
        \node[anchor=south, font=\bfseries\color{black!90!black}] at (0.7, 0.4) {$H$};
        
        \node[cella] (c0) at (0.7, 0) {}; \node[indice] at (c0.west) {$0$};
        \node[cella, below=0pt of c0] (c1) {}; \node[indice] at (c1.west) {$1$};
        \node[cella, below=0pt of c1] (c2) {}; \node[indice] at (c2.west) {$2$};
        
        \node[cella, below=0pt of c2, fill=black!2] (cdot) {}; 
        % \node at (cdot.center) {$\vdots$};
        
        \node[cella, below=0pt of cdot] (ci) {}; \node[indice] at (ci.west) {$i$};
        
        \node[cella, below=0pt of ci, fill=black!2] (cdot2) {};
        % \node at (cdot2.center) {$\vdots$};
        
        \node[cella, below=0pt of cdot2] (cm) {}; \node[indice] at (cm.west) {$m-1$};
    \end{scope}

    % --- Liste di Trabocco (Chaining) ---
    % Lista per l'indice 1 (Collisione k1, k4)
    \node[lista, right=0.5 of c1] (l1_1) {$k_1$};
    \node[lista, right=.5 of l1_1] (l1_2) {$k_4$};
    % \node[right=0.2 of l1_2] (null1) {$\slash$};
    
    % Lista per l'indice i (Solo k2)
    \node[lista, right=0.5 of ci] (li_1) {$k_2$};
    % \node[right=0.2 of li_1] (null2) {$\slash$};

    % Lista per l'indice m-1 (Solo k3)
    \node[lista, right=0.5 of cm] (lm_1) {$k_3$};
    % \node[right=0.2 of lm_1] (null3) {$\slash$};
    
    % Disegno puntatori liste
    \draw[puntamento] (c1.center) -- (l1_1.west);
    \draw[puntamento] (l1_1.east) -- (l1_2.west);
    \draw[puntamento] (ci.center) -- (li_1.west);
    \draw[puntamento] (cm.center) -- (lm_1.west);

    % --- Mapping della Funzione di Hash ---
    % Le frecce puntano verso l'indice, non toccano la cella (shorten >)
    \draw[funzione, shorten >=15pt] (k1) to[out=15, in=180] node[midway, below, sloped, black] {$h(k_1)$} (c1.west);
    % \draw[funzione, shorten >=15pt] (k4) to[out=15, in=180] node[midway, above, sloped, black] {$h(k_1)$} (c1.west);
    \draw[funzione, dashed, blue!40, shorten >=15pt] (k4) to[out=0, in=160] node[midway, above, sloped, blue!70] {$h(k_4)$} (c1.west);
    
    \draw[funzione, shorten >=15pt] (k2) to[out=0, in=180] node[midway, below, sloped, black] {$h(k_2)$} (ci.west);
    \draw[funzione, shorten >=33pt] (k3) to[out=-15, in=180, left=1] node[midway, below, sloped, black] {$h(k_3)$} (cm);

    % --- Annotazioni Finali ---
    \node[align=center, text width=7cm, font=\itshape, color=black!90!black] at (3.5, -4.8) {
        Risoluzione delle collisioni mediante \\ \textbf{liste di trabocco}.
        };

    \end{tikzpicture}
\end{figure}

Le operazioni sono implementate come segue, 
viene calcolato il valore di $h(u)$, successivamente l'operazione viene eseguita scandendo $H(h(u))$.
È quindi necessario utilizzare una funzione hash $h$ che distribuisca gli elementi di $S$ in modo uniforme

\vspace{1em}\noindent
È possibile utilizzare la tecnica dell' \emph{hashing deterministico}, in cui ogni elemento $u \in U$ è rappresentato come intero, scegliendo un primo $p$, tale che $m \leq p \leq 2m$ allora $h$ è definito come 
\[
    h(u) = u \mod p
\]
funziona bene per applicazioni statiche in cui $S$ non varia.

\vspace{1em}
Per applicazioni in cui $S$ varia si fa uso delle funzioni hash universali, di cui sono state precedentemente descritte le proprietà ed esempi.
Si ricordando le variabili $n$ rappresentate il numero di elementi di $S$, $N = |U|$, $m$ numero primo compreso $N \leq m \leq 2N$. Si fa uso della tecnica di \emph{doubling-halving}

\begin{algorithm}[H]
    \caption{Doubling/Halving Technique}
    \DontPrintSemicolon
    \SetKwComment{Comment}{// }{}

    % \KwInit{$n = N = 1$}

    \BlankLine
    \Comment{In caso di espansione}
    \If{$n > N$}{
        $N \gets 2N$\;
        scegli un nuovo numero primo $m$ tale che $m \sim \Theta(n)$\;
        re-hash di tutti gli elementi in $O(n)$\;
    }

    \BlankLine
    \Comment{In caso di contrazione}
    \If{$n < N/4$}{
        $N \gets N/2$\;
        scegli un nuovo numero primo $m$ tale che $m \sim \Theta(n)$\;
        re-hash di tutti gli elementi in $O(n)$\;
    }
\end{algorithm}
Si ottiene tempo $O(1)$ ammortizzato per eliminazione e inserimento, poiché il re-hashing avviene solo quando la dimensione raddoppia.

%TODO: Fare perfect hashing
\end{document}